\chapter{Computing Galois groups of polynomials}

\section{A specific quintic not solvable by radicals}

\subsection{Preliminaries}

\begin{theorem}[Cauchy]
    If $G$ is a finite group with order has a prime divisor $p$, then $G$ contains an element of order $p$.
\end{theorem}

\begin{proof}
    Consider the set
    \[
        X := \{(g_1, g_2, \ldots, g_p) \in G^p: g_1 \cdots g_p = 1\}.
    \]
    The last component $g_p$ in any element of $X$ is completely determined by the other $p - 1$ components, so $|X| = |G|^{p - 1}$, which is divisible by $p$. Also, $X$ is closed under cyclic permutations, since
    \[
        g_1 \cdots g_p = 1 \implies g_1 \cdots g_p g_p^{-1} = g_p^{-1} \implies g_p g_1 \cdots g_{p - 1} = 1.
    \]
    Now, let $C_p = \zn{p}$ act on $X$ via cyclic permutation, i.e.
    \[
        \overline{k} \, . \, (g_1, \ldots, g_p) = (g_k, g_{k + 1}, \ldots, g_p, g_1, \ldots, g_{k - 1})
    \]
    for each $\overline{k} \in C_p$. Since $p$ is prime, each orbit contains either 1 or $p$ elements by the orbit-stabiliser theorem. Also, it is easy to see that the only single-element orbits are those of tuples $(g, g, \ldots, g)$, and each is stabilised by $C_p$. All other orbits have trivial stabiliser. If there are $n$ single-element orbits, then by Burnside's lemma we have that
    \[
       |X/C_p| = \frac{1}{|C_p|} \sum_{x \in X} \left|\stab{C_p}{x}\right| \; \implies \; p |X/C_p| = np + |X| - n.
    \]
    Reducing modulo $p$, we see that $p$ divides $n$, and clearly $n \neq 0$, since $(1, 1, \ldots, 1)$ has single-element orbit. But this implies that $n \geq p$, so in particular there must exist $g \in G$ with $g^p = 1$.
    % Consider the equivalence classes of $X$ under cyclic permutations. That is, we consider two tuples in $X$ equivalent if one is obtained by cyclic permutation of the other. Each equivalence class is either of the form
    % \begin{enumerate}
    %     \item $[(g, g, \ldots, g)]$, of size 1, or
    %     \item $[(g_1, g_2, \ldots, g_p)]$, where the $g_k$ are not all equal, of size $p$.
    % \end{enumerate}
    % The number of equivalence classes of the first form must be divisible by $p$ (why?), and there is at least one class of this type, namely $[(1, \ldots, 1)]$, so there must be another one, say $[(g, \ldots, g)]$, i.e. $g^p = 1$ for some $g \in G$.
\end{proof}

\begin{remark}
    This is like a partial converse to Lagrange's theorem, but may fail if $p$ is not prime. For example, $6 \mid |A_4|$, but $A_4$ has no subgroup of order 6.
\end{remark}

\begin{lemma}
    Let $p$ be a prime, and $\tau, \sigma \in S_p$ where $\tau$ is a 2-cycle and $\sigma$ a $p$-cycle. Then $S_p = \gen{\tau, \sigma}$.
\end{lemma}

\begin{proof}
    By relabelling indices, we may assume $\tau = (12)$. By choosing $k$ such that $\sigma^k(1) = 2$ and then relabelling further indices, we may also assume $\sigma = (1234 \cdots p)$. Notice that since $\sigma^j \tau \sigma^{-j} = (j + 1 \; j + 2)$, we can obtain any simple transposition as a product from $\sigma$ and $\tau$. The set of all such transpositions generates $S_p$, so $S_p = \gen{\tau, \sigma}$.
\end{proof}

\subsection{Insolvable polynomials}

\begin{proposition}
    Let $f \in \bb{Q}[x]$ be an irreducible polynomial of prime degree $p$ with exactly 2 non-real roots. Then $\gal(f/\bb{Q}) \cong S_p$.
\end{proposition}

\begin{proof}
    Let $K$ be the splitting field of $f$ over $\bb{Q}$. We can identify $G = \gal(K/\bb{Q})$ with a subgroup of $S_p$. The complex conjugation map $\tau$ is in $G$, and it must correspond to a 2-cycle in this embedding, since $f$ has exactly 2 non-real roots. Let $\alpha$ be any root of $f$. Then because $f$ is irreducible, we have $[\bb{Q}(\alpha) : \bb{Q}] = p$, so by the tower law, $[K : \bb{Q}] = |G|$ is divisible by $p$. Cauchy's theorem now implies that $G$ contains an element $\sigma$ of order $p$. In particular, $\sigma$ must be a $p$-cycle, so $G \cong S_p$ by the previous lemma, as it contains a 2-cycle and a $p$-cycle.
\end{proof}

\begin{remark}
    In particular, if $p \geq 5$, then $f$ is not solvable by radicals.
\end{remark}

\begin{example}~
    \begin{enumerate}
        \item Let us show that $f(x) = x^5 - 6x + 3$ is insolvable. It is certainly irreducible by Eisenstein's criterion. Showing that $f$ has exactly two non-real roots is left as a calculus exercise, but by the previous proposition, we now know $\gal(f/\bb{Q}) \cong S_5$, which is not solvable.
        
        \item For a non-example showing where this proposition fails if the degree is not prime, consider $f(x) = x^4 - 2x^2 - 1$. This is irreducible over $\bb{Q}$, and has exactly two non-real roots, but $G \cong D_8 \neq S_4$.
    \end{enumerate}
\end{example}

\section{The discriminant of a polynomial}

Let $f \in F[x]$ be of degree $n$, and $G = \gal(f/F)$. Viewing $G$ as a subgroup of $S_n$, we can consider the normal subgroup $N := G \cap A_n$ of $G$. By the second isomorphism theorem, we have
\[
    G/N \cong GA_n/A_n \leq S_n/A_n \cong \zn{2},
\]
so there is an inclusion $G/N \inclusion \zn{2}$. Moreover, this inclusion is an isomorphism if and only if $G \not\leq A_n$\footnote{Unresolved: is this subgroup or subset? Also, why is any of this relevant?}.

Let us reflect for a moment on an alternate definition of what it means for a permutation to be odd or even. Consider the polynomial
\[
    \Delta(x_1, \ldots, x_n) = \prod_{k < j} (x_k - x_j) \in \bb{Z}[x_1, \ldots, x_n],
\]
and take any $\sigma \in S_n$. Then
\[
    (\sigma \Delta)(x_1, \ldots, x_n) = \Delta(x_{\sigma(1)}, \ldots, x_{\sigma(n)}) = \begin{cases}
        \Delta & \text{if } \sigma \text{ is even}, \\
        -\Delta & \text{if } \sigma \text{ is odd}.
    \end{cases}
\]

\begin{proposition}
    Suppose that $\fieldchar{F} \neq 2$, and $\alpha_1, \ldots, \alpha_n \in K$ are distinct roots of $f$ in a splitting field $K$ for $f$ over $F$. Then
    \begin{enumerate}[label=(\alph*)]
        \item $K^N = F(\delta)$ where $\delta := \prod_{k < j} (\alpha_k - \alpha_j)$.
        \item $D := \delta^2 \in F$.
        \item $G \leq A_n$ if and only if $D$ is already a square in $F$.
    \end{enumerate}
\end{proposition}

\begin{proof}~
    \begin{enumerate}[label=(\alph*)]
        \item The reminder above implies that for any $\sigma \in G$, $\sigma(\delta) = \delta$ if $\sigma$ is even, or $\sigma(\delta) = -\delta$ if $\sigma$ is odd. That is, $\delta$ is in the fixed field $K^\sigma$ if and only if $\sigma \in N$, i.e. $\gal(K/F(\delta)) = N$. So by the Galois correspondence, $K^N = F(\delta)$.
        \item It is enough to prove that $\sigma(D) = D$ for all $\sigma \in G$, and indeed
        \[
            \sigma(D) = \sigma(\delta)^2 = (\sigma(\delta))^2 = (\pm \delta)^2 = D.
        \]
        So $D \in K^G = F$.
        
        \item Note that $G \leq A_n$ if and only if $N = G$. In turn, by the Galois correspondence, this holds if and only if $F(\delta) = F$. \qedhere
    \end{enumerate}
\end{proof}

\begin{definition}
    We call the quantity $D$ above the \emph{discriminant} of $f$.
\end{definition}

This discriminant can also be expressed in terms of the coefficients of $f$.

\begin{example}
    In the case of a monic quadratic $f(x) = x^2 + bx + c$, we have
    \[
        D = (\alpha_1 - \alpha_2)^2 = \alpha_1^2 - 2\alpha_1 \alpha_2 + \alpha_2^2 = (\alpha_1 + \alpha_2)^2 - 4 \alpha_1 \alpha_2 = b^2 - 4c.
    \]
\end{example}

\begin{example}
    Consider the depressed cubic $f(x) = x^3 + px + q$. One can show that $D = -27q^2 - 4p^3$. As a concrete example, for $x^3 + 2x + 2$, we have $D < 0$, so $D$ is not a square in $\bb{Q}$. This implies that $\gal(f/\bb{Q}) \cong S_3$. Considering now a non-depressed cubic $f(x) = x^3 + x^2 - 2x - 1$, we saw in tutorial 2 that if $\alpha$ was a root, then so is $\alpha^2 - 2$. So the splitting field is actually $\bb{Q}(\alpha)$. With calculation, one sees that $D = 49$, so in this case $\gal(f/\bb{Q}) \cong A_3$.
\end{example}
