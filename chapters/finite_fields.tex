\chapter{Galois theory for finite fields}

\section{Galois groups of finite fields}

Recall that if $K$ is a finite field, then
\begin{enumerate}[label=(\alph*)]
    \item $\fieldchar{K} = p$ for some prime $p$.
    \item $K$ contains a unique copy of $\bb{F}_p$ as a subfield.
    \item $|K| = p^k$, where $d = [K : \bb{F}_p]$.
\end{enumerate}
We now briefly discuss how Galois theory works for these fields.

\begin{proposition}
    Let $k \in \bb{Z}^{+}$. Then
    \begin{enumerate}[label=(\alph*)]
        \item The splitting field $K$ of $x^{p^k} - x$ over $\bb{F}_p$ is a finite field of order $p^k$. (In particular, $\bb{F}_{p^k}$ exists.)
        \item Any field $L$ with $|L| = p^k$ is isomorphic to $K$. (That is, $\bb{F}_{p^k}$ is unique up to isomorphism.)
        \item The extension $K/\bb{F}_p$ is Galois with $\gal(K/\bb{F}_p) = \gen{\varphi} \cong \zn{k}$, where $\varphi: K \to K$ is the $p$th power Frobenius endomorphism of $K$. 
    \end{enumerate}
\end{proposition}

\begin{proof}~
    \begin{enumerate}[label=(\alph*)]
        \item Let $f(x) = x^{p^k} - x$. By \cref{prop_separability_conditions}, $f$ has $p^k$ distinct roots, since
        \[
            f'(x) = p^k x^{p^{k - 1}} - 1 = -1 \neq 0
        \]
        in $\bb{F}_p$. So $|K| \geq p^k$. Now consider the automorphism $\varphi^k: x \mapsto x^{p^k}$ and its fixed field. We have
        \begin{align*}
            \alpha \in K^{\varphi^k}
            &\iff \varphi^k(\alpha) = \alpha \\
            &\iff \alpha^{p^k} - \alpha = 0 \\
            &\iff \alpha \text{ is a root of } f,
        \end{align*}
        so the roots of $f$ are already a field, namely $K^{\varphi^k}$. This shows that $K = K^{\varphi^k}$, and also that $\varphi^k = \id_K$.
        
        \item Let $|L| = p^k$. By \cref{thm_exist_uniq_of_spl_field}, if we can show that $L$ is a splitting field of $x^{p^k} - x$ over $\bb{F}_p$, then we are done. By Lagrange's theorem, if $\alpha \in L^*$, then $\alpha^{p^k - 1} = 1$, i.e. $\alpha^{p^k} = \alpha$. So $x^{p^k} - x$ factors as
        \[
            x^{p^k} - x = (x - 0) \prod_{\alpha \in L^*} (x - \alpha) = \prod_{\beta \in L} (x - \beta),
        \]
        over $L$, and clearly $L$ is generated by itself.
        
        \item By (a), we have $\varphi^k = \id_K$. Consider now the fixed field
        \[
            K^{\varphi} = \{\alpha \in K: \alpha^p = \alpha\} = \{\text{roots of } x^p - x\} = \bb{F}_p.
        \]
        So $K/\bb{F}_p$ is Galois with $\gal(K/\bb{F}_p) = \gen{\varphi}$. Since $|G| = [K : \bb{F}_p] = k$, this forces $\ord{\varphi} = k$, i.e. $G \cong \zn{k}$. \qedhere
    \end{enumerate}
\end{proof}

\begin{example}
    Let us find all of the subfields of $K = \bb{F}_{p^{12}}$. By the previous proposition, we know $\gal(K/\bb{F}_p) \cong \zn{12}$, so we can use the Galois correspondence to work out the subfield structure of $K$:
    \[
        \begin{tikzcd}
        	& {\mathbb{F}_{p^{12}}} &&& {12\mathbb{Z}/12\mathbb{Z}} \\
        	{\mathbb{F}_{p^6}} && {\mathbb{F}_{p^4}} & {6\mathbb{Z}/12\mathbb{Z}} && {4\mathbb{Z}/12\mathbb{Z}} \\
        	{\mathbb{F}_{p^3}} && {\mathbb{F}_{p^2}} & {3\mathbb{Z}/12\mathbb{Z}} && {2\mathbb{Z}/12\mathbb{Z}} \\
        	& {\mathbb{F}_p} &&& {\mathbb{Z}/12\mathbb{Z}}
        	\arrow["3", no head, from=4-2, to=3-1]
        	\arrow["2"', no head, from=4-2, to=3-3]
        	\arrow["2"', no head, from=3-3, to=2-3]
        	\arrow["2", no head, from=3-1, to=2-1]
        	\arrow["2", no head, from=2-1, to=1-2]
        	\arrow["3"', no head, from=2-3, to=1-2]
        	\arrow["3"', no head, from=3-3, to=2-1]
        	\arrow["2"', no head, from=4-5, to=3-6]
        	\arrow["2"', no head, from=3-6, to=2-6]
        	\arrow["3"', no head, from=2-6, to=1-5]
        	\arrow["2", no head, from=2-4, to=1-5]
        	\arrow["2", no head, from=3-4, to=2-4]
        	\arrow["3", no head, from=4-5, to=3-4]
        	\arrow["3"', no head, from=3-6, to=2-4]
        \end{tikzcd}
    \]
\end{example}

\begin{remark}
    While the subgroups of the Galois group differ for each power $k$, the overall structure of the Galois correspondence for $\bb{F}_{p^k}/\bb{F}_p$ is similar.
\end{remark}

\section{Bonus: the algebraic closure of $\bb{F}_p$ (TODO)}

\begin{block}
    This section is unfinished. Notes have been taken, but this wasn't assessable material, so there wasn't much incentive for me to ensure this had solid coverage.
\end{block}

% Assuming Zorn's lemma (which is equivalent to the axiom of choice), one can prove that any field $F$ has an algebraic closure $\overline{F}$. As it turns out, though, $\overline{\bb{F}}_p$ may be ``explicitly" constructed:
% \begin{enumerate}
%     \item Let $K_1 = \bb{F}_p$.
%     \item Choose any irreducible, quadratic polynomial in $\bb{F}_p[x]$\footnote{To be constructive, list all of these polynomials in lexicographic order and pick the first one.}, and adjoin a root $\alpha_2$ to $K_1$ to obtain the field $K_2 = K_1(\alpha_2)$.
%     \item Choose any irreducible, cubic polynomial in $\bb{F}_p[x]$. This cubic will also be irreducible over $K_2$, so adjoin a root $\alpha_3$ to $K_2$ to obtain the field $K_3 = K_2(\alpha_3)$.
%     \item Choose any irreducible, quartic polynomial in $\bb{F}_p[x]$. One can show that this polynomial will not be irreducible over $K_3$, because $\bb{F}_{p^2} \subseteq K_3$. It will instead factor into two irreducible quadratic polynomials, so we adjoin a root $\alpha_4$ of one of these polynomials\footnote{Again, to be constructive, just take the lexicographically smallest.} to obtain $K_4 = K_3(\alpha_4)$.
%     \item Continuing, we get $\bb{F}_{p^5} \subseteq K_5$. We can form $K_6 = K_5$, since $\bb{F}_{p^2}, \bb{F}_{p^3} \subseteq K_5$ implies that $\bb{F}_{p^6} \subseteq K_5$ already. The process now continues indefinitely in this way.
% \end{enumerate}

% At the end, we obtain an infinite tower
% \[
%     \bb{F}_p = K_1 \subseteq K_2 \subseteq K_3 \subseteq \cdots,
% \]
% and we have explicit defining irreducible polynomials at each step. The algebraic closure is then the union
% \[
%     \overline{\bb{F}}_p = \bigcup_{j = 1}^{\infty} K_j,
% \]
% though some care is required with what this means, and the most helpful way to view it is as a disjoint union modulo equivalence classes. Explicitly, if $\beta \in K_j$, then we can declare an equivalence relation $\equiv_\beta$ on all other fields $K_m$ for which $\beta \in K_{m}$ as equivalent to $K_j$; as a concrete representative, one can then choose the minimal such $j$. We can
