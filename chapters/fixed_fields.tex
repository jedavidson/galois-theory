\chapter{Field automorphisms and fixed fields}

\section{Field automorphisms}

In all that follows, let $K$ be some field.

\begin{definition}
    A \emph{field automorphism} of $K$ is a ring isomorphism $\sigma: K \to K$.
\end{definition}

\begin{example}
\label{exmp_field_auts}
    \begin{enumerate}
        \item The identity map is a trivial field automorphism.
        \item The complex conjugation map $\sigma : z \longmapsto \overline{z}$ is a field automorphism of $\bb{C}$.
        \item Let $K = \bb{Q}(\sqrt{2}, \sqrt{3}) = \bb{Q}(\sqrt{2})(\sqrt{3})$. A typical element of $K$ is
        \[
            u_1 + u_2 \sqrt{2} + u_3 \sqrt{3} + u_4 \sqrt{6}
        \]
        for some $u_k \in \bb{Q}$. Define $\tau: K \to K$ by
        \[
            u_1 + u_2 \sqrt{2} + u_3 \sqrt{3} + u_4 \sqrt{6} \longmapsto u_1 + u_2 \sqrt{2} - u_3 \sqrt{3} - u_4 \sqrt{6}.
        \]
        Seeing that this is a field automorphism is arduous by definition. The proof is an exercise, but matters can be made simpler if we first write
        \[
            v_1 = u_1 + u_2\sqrt{2}
            \quad \text{and} \quad
            v_2 = u_3 + u_4\sqrt{2},
        \]
        where to be clear $v_1, v_2 \in \bb{Q}(\sqrt{2})$, so that
        \[
            v_1 + v_2 \sqrt{3} \longmapsto v_1 - v_2 \sqrt{3}.
        \]
        \item Considering $K$ from the previous example again, we can similarly look at
        \[
            \sigma: u_1 + u_2 \sqrt{2} + u_3 \sqrt{3} + u_4 \sqrt{6} \longmapsto u_1 - u_2 \sqrt{2} + u_3 \sqrt{3} - u_4 \sqrt{6},
        \]
        which turns out to be another field automorphism.
        \item $\sigma \tau$ is also a field automorphism. (What does it do to elements of $K$?)
        \item The map
        \[
            u_1 + u_2 \sqrt{2} + u_3 \sqrt{3} + u_4 \sqrt{6} \longmapsto u_1 + u_2 \sqrt{2} + u_3 \sqrt{3} - u_4 \sqrt{6}.
        \]
        is \textbf{not} a field automorphism. (Why not?)
    \end{enumerate}
\end{example}

\begin{proposition}
    The set $\aut{K}$ of field automorphisms of $K$ is a subgroup of $\perm{K}$ (i.e. the set of all bijections $K \to K$).
\end{proposition}

\begin{proof}
    See the week 1 tutorial problems.
\end{proof}

\begin{example}
    We discuss some examples and non-examples of field automorphisms, this time of $K = \bb{C}(x)$.
    \begin{enumerate}
        \item The maps $g(x) \longmapsto g(\alpha x)$ and $g(x) \longmapsto g(x + \alpha)$ are field automorphisms for any nonzero $\alpha \in \bb{C}$. More generally, maps of the form
        \[
            g(x) \longmapsto g\left(\frac{ax + b}{cx + d}\right)
        \]
        for $ad - bc \neq 0$ are field automorphisms.
        
        \item The map $g(x) \longmapsto \overline{g}(x)$ is a field automorphism, interpreting $\overline{g}$ as having coefficients conjugate to the respective coefficients of $g$.
        
        \item The map $g(x) \longmapsto \alpha g(x)$ for $z \in \bb{C} \setminus \{1\}$ is not a field automorphism: it does not preserve the identity map of $K$.

        \item Neither is $g(x) \longmapsto g(x^2)$, despite being a field homomorphism, because it is not surjective: every image by this morphism is an even function.
    \end{enumerate}
\end{example}

\section{Fixed fields}

\begin{definition}
    Let $S \subseteq \aut{K}$. The \emph{fixed field} of $S$ is
    \[
        K^S = \{\alpha \in K: \sigma(\alpha) = \alpha \text{ for all } \sigma \in S\}.
    \]
\end{definition}

\begin{proposition}
    $K^S$ is a subfield of $K$.
\end{proposition}

\begin{proof}
    See the week 1 tutorial problems.
\end{proof}

\begin{example}
    Let $K = \bb{C}$ and $\sigma$ the conjugation automorphism as discussed in \cref{exmp_field_auts}. Then $\bb{C}^{\sigma} = \bb{C}^{\{\sigma\}} = \bb{R}$.
\end{example}

\begin{proposition}
    Some nice ``composite'' properties of fixed fields.
    \begin{enumerate}[label=(\alph*)]
        \item If $S' \subseteq S \subseteq \aut{K}$, then $K^{S} \subseteq K^{S'}$. (The fixed field inclusions are flipped!)
        \item We have $K^{\gen{S}} = K^{S}$, where $\gen{S}$ is the subgroup of $\aut{K}$ generated by $S$.
    \end{enumerate}
\end{proposition}

\begin{proof}
    See the week 1 tutorial problems.
\end{proof}

\begin{example}
    Consider the automorphisms $\tau, \sigma$ of $K = \bb{Q}(\sqrt{2}, \sqrt{3})$ from \cref{exmp_field_auts}. It is not hard to see that $K^\tau = \bb{Q}(\sqrt{2})$ and $K^{\gen{\sigma, \tau}} = \bb{Q}$. Experiment with some other automorphism sets and work out their fixed fields.
\end{example}

\begin{example}
    Let $K = \bb{C}(x)$ and $\sigma: g(x) \longmapsto g(-x)$. Then $K^\sigma = \bb{C}(x^2)$, i.e. the even maps. Proving this is an exercise in the week 1 tutorial problems.
\end{example}

\section{An application: minimal polynomials via symmetry}

\begin{theorem}
\label{thm_aut_min_polys}
    Let $G$ be a \emph{finite} subgroup of $\aut{K}$, $\alpha \in K$ and $\{\alpha_1, \alpha_2, \ldots, \alpha_n\}$ be the orbit of $\alpha$ under $G$ (i.e. all images of $\alpha$ under automorphisms in $G$). Then
    \[
        p(x) := \prod_{i = 1}^{n} (x - \alpha_i)
    \]
    is the minimal polynomial of $\alpha$ over $K^G$.
\end{theorem}

Before giving the proof, we first give two examples and then a proposition of rings that we will need in the proof.

\begin{example} \cref{thm_aut_min_polys} in action.
    \hphantom{aaa}
    \begin{enumerate}
        \item Let $K = \bb{C}$, $G = \gen{\sigma} = \{1, \sigma\}$, where $\sigma$ is the conjugation map. We saw that $K^\sigma = \bb{R}$. If $\alpha \in \bb{R}$, then the orbit of $\alpha$ under $G$ is just $\alpha$, so the minimal polynomial over $K^G = \bb{R}$ is $x - \alpha$. On the other hand, if $\alpha \notin \bb{R}$, then the orbit is $\{\alpha, \overline{\alpha}\}$, and so the minimal polynomial is $(x - \alpha)(x - \overline{\alpha})$. So the theorem seemingly generalises something we already knew about $\bb{C}$.
        
        \item Consider again $K = \bb{Q}(\sqrt{2}, \sqrt{3})$ and the automorphisms $\sigma, \tau$ of \cref{exmp_field_auts}. Let $G = \gen{\sigma, \tau} = \{1, \sigma, \tau, \sigma \tau\}$ and $\alpha = 3 + 2\sqrt{2} - \sqrt{3}$. The orbit of $\alpha$ has 4 elements, giving a quartic minimal polynomial of $\alpha$ over $K^G = \bb{Q}$. Another way to calculate the minimal polynomial here is to try candidate minimal polynomials of increasing degree, but this is tedious (it amounts to trying to see if one can solve systems of equations over $\bb{Q}$).
    \end{enumerate}
\end{example}

\begin{proposition}
\label{prop_ring_hom_to_poly_ring_hom}
    Let $\varphi: R \to S$ be a ring homomorphism. Then $\varphi$ induces a ring homomorphism $R[x] \to S[x]$.
\end{proposition}

\begin{proof}
    Tutorial exercise. (The idea is to apply $\varphi$ to each coefficient of $R[x]$.)
\end{proof}

\begin{remark}
    We will use $\varphi$ for the name of this induced homomorphism too, though the literature may give it a different name (e.g. $\widetilde{\varphi}$ or $\varphi[x]$).
\end{remark}

\begin{proof}[Proof of \cref{thm_aut_min_polys}]
    One of the $\alpha_i$'s must be $\alpha$, so $p(\alpha) = 0$. We first check that $p \in K^G[x]$ (i.e. its coefficients are fixed by $G$). Take any $\sigma \in G$. Using \cref{prop_ring_hom_to_poly_ring_hom}, $\sigma$ lifts to a homomorphism $K^G[x] \to K^G[x]$, which in effect applies $\sigma$ to $p$ factor-by-factor so that
    \[
        (\sigma p)(x) = \prod_{i = 1}^{n} \sigma(x - \alpha_i) = \prod_{i = 1}^{n} (x - \sigma(\alpha_i)) = \prod_{i = 1}^{n} (x - \alpha_i) = p(x),
    \]
    where the second-to-last inequality holds because $\sigma$ permutes the orbit. So $p \in K^G[x]$. Now, if $q$ is the monic minimal polynomial of $\alpha$ over $K^G[x]$, then by definition $q \mid p$ (in $K^G[x]$) and $(x - \alpha) \mid q(x)$ (in $K[x]$). Also, since every $\sigma \in G$ is a homomorphism, we have $\sigma(x - \alpha) \mid (\sigma q)(x)$ (in $K[x]$), i.e. $x - \sigma(\alpha) \mid q(x)$ (since $q$ is fixed by $\sigma$). By choosing different automorphisms $\sigma$ we can obtain each $\alpha_i$ in place of $\sigma(\alpha)$, so $(x - \alpha_i) \mid q(x)$ for all $i = 1, \ldots, n$. The product of these linear factors, which is precisely $p$, therefore divides $q$, so in fact $p = q$.
\end{proof}
