\chapter{Galois closure}

We have seen that not all extensions we might want to study are Galois, as they ``lack roots". The aim for this chapter is to outline a way in which any separable extension may be embedded in an extension which \emph{is} Galois, which then allows us to do meaningful Galois theory with it.

\section{Definitions}

In what follows, let $K/F$ be a finite, separable (but not necessarily Galois) extension generated by separable $\alpha_1, \ldots, \alpha_n \in F$. Let $p_1, \ldots, p_n$ be the minimal polynomials of these generators over $F$.

\begin{definition}
    A \emph{Galois closure} for $K/F$ is any splitting field $L$ for $p = p_1 \cdots p_n$ over $K$.
\end{definition}

\begin{example}
    We noted quite early on that $\bb{Q}(\sqrt[3]{2})/\bb{Q}$ was not Galois. The minimal polynomial for its single generator $\alpha_1 = \sqrt[3]{2}$ is $p_1(x) = x^3 - 2$, and the splitting field for $p_1$ is $\bb{Q}(\sqrt[3]{2}, \zeta_3)$, which is a Galois closure of $\bb{Q}(\sqrt[3]{2})/\bb{Q}$.
\end{example}

What is perhaps unclear at this stage is why this definition is independent of the choice of generators $\alpha_k$. We will address this in the following proposition:

\begin{proposition}
    Let $L$ be a Galois closure of $K/F$. Then
    \begin{enumerate}[label=(\alph*)]
        \item $L/F$ is a finite Galois extension.
        \item Let $\widetilde{K}/K$ be an extension such that $\widetilde{K}/F$ is Galois. Then there is an $F$-homomorphism $\varphi: L \to \widetilde{K}$. (In other words, the Galois closure embeds into any Galois extension.)
        \item In particular, Galois closures are unique up to $F$-isomorphism.
    \end{enumerate}
\end{proposition}

\begin{proof}~
    \begin{enumerate}[label=(\alph*)]
        \item Clear by definition.
        \item Take $L$ as a splitting field of $p$ as above. By \cref{thm_aut_min_polys}, $p$ factors linearly over $\widetilde{K}$, so $\widetilde{K}$ contains a splitting field $\widetilde{L}$ for $p$ over $F$. Any two splitting fields over $F$ are isomorphic by \cref{thm_exist_uniq_of_spl_field}, so there is an isomorphism $\varphi: L \to \widetilde{L}$. In particular, we may regard $\varphi$ as an injective homomorphism $L \to \widetilde{K}$, and $\varphi$ must also fix $F$ as noted in \cref{rem_spl_field_isom_fixes_base}.
        \item Let $L, \widetilde{L}$ be Galois closures of $K/F$. By part (b), there is an injective field homomorphism $\varphi: L \to \widetilde{L}$. Applying the argument in (b) in the other direction, there is also an injective field homomorphism $\psi: \widetilde{L} \to L$. So we have $[L : F] \leq [\widetilde{L} : F]$ and $[\widetilde{L} : F] \leq [L : F]$, which implies both $\varphi, \psi$ are $F$-isomorphisms. \qedhere
    \end{enumerate}
\end{proof}

\begin{remark}
    One potential problem with the Galois closure in practice, potentially with respect to the computational side of things, is that $[L : F]$ may be quite large relative to $[K : F]$.
\end{remark}

\section{Galois closures of radical extensions}

\begin{proposition}
\label{prop_gal_closure_of_rad_ext}
    The Galois closure of a radical extension $K/F$ is radical.
\end{proposition}

\begin{proof}
    Suppose that $K/F$ has a radical tower
    \[
        F \subseteq F(\alpha_1) \subseteq F(\alpha_1, \alpha_2) \subseteq F(\alpha_1, \ldots, \alpha_n) = K.
    \]
    where each $\alpha_k^{m_k} \in F(\alpha_1, \ldots, \alpha_{k - 1})$ for integers $m_k \geq 1$. Let $L$ be the Galois closure of $K/F$, and let
    \[
        \gal(L/F) = \{\sigma_1, \ldots, \sigma_m\}.
    \]
    We must show that there is a radical tower up to $L$. To start, consider the tower
    \[
        F \subseteq F(\alpha_1) \subseteq F(\alpha_1, \sigma_1(\alpha_1)).
    \]
    Because $\alpha_1^{m_1} \in F$, we also have $\sigma_1(\alpha_1^{m_1}) \in F$, because $\sigma_1$ fixes $F$. Repeating for the other automorphisms $\sigma_k$, we obtain the tower
    \begin{align*}
        F
        \subseteq F(\alpha_1)
        \subseteq F(\alpha_1, \sigma_1(\alpha_1))
        &\subseteq F(\alpha_1, \sigma_1(\alpha_1), \sigma_2(\alpha_1)) \\
        &\subseteq \cdots \\
        &\subseteq F(\alpha_1, \sigma_1(\alpha_1), \ldots, \sigma_m(\alpha_1)).
    \end{align*}
    We can continue this radical tower up to $L$ by successively adjoining $\alpha_2$ and all of its images under the $\sigma_k$'s, and similarly for $\alpha_3, \ldots, \alpha_n$.
\end{proof}

\begin{remark}
    The radical tower produced here may be an overapproximation: after all, we are adjoining $(m + 1) \cdot n$ roots!
\end{remark}

\section{The primitive element theorem}

\begin{definition}
    Let $K/F$ be a field extension. If $K$ is generated over $F$ by $\alpha \in K$, we say that $\alpha$ is a \emph{primitive element} for $K/F$.
\end{definition}

\begin{theorem}[Primitive element theorem]
\label{thm_prim_elem}
    Every finite, separable extension has a primitive element.
\end{theorem}

\begin{proof}
    First suppose that $F$ is finite. Then $K$ is also finite, and $K^*$ is cyclic. So it suffices to pick any generator of $K^*$ as our primitive element.
    
    Now assume $F$ is infinite, and suppose toward a contradiction that $K \neq F(\alpha)$ for any $\alpha \in K$. Pick any $\alpha \in K$ and some $\beta \in K \setminus F(\alpha)$. Let $L$ be the Galois closure of $K/F$ (which we may take since $K/F$ is separable). By the fundamental theorem, $L/F$ has only finitely-many intermediate fields, so $K/F$ also has only finitely-many intermediate fields. Considering those of the form $F(\alpha + c\beta)$ where $c \in F$, the pigeonhole principle implies that there are distinct $c_1, c_2 \in F$ with $F(\alpha + c_1\beta) = F(\alpha + c_2\beta)$. Since
    \[
        \beta = \frac{(\alpha + c_1 \beta) - (\alpha + c_2 \beta)}{c_1 - c_2} \in F(\alpha + c_1 \beta),
    \]
    we have $\alpha \in F(\alpha + c_1 \beta)$. So $F(\alpha) \subsetneq F(\alpha + c_1 \beta)$. Because $K/F(\alpha + c_1 \beta)$ is also a finite, separable extension, we can inductively construct an arbitrary number of intermediate fields in this way, which is a contradiction.
\end{proof}

\begin{remark}
    This theorem is false in general for inseparable extensions.
\end{remark}

\begin{example}
    A primitive element for $\bb{Q}(\sqrt{2}, \sqrt{3})/\bb{Q}$ is $\alpha = \sqrt{2} + \sqrt{3}$, but anything not in $\bb{Q}(\sqrt{2})$, $\bb{Q}(\sqrt{3})$ or $\bb{Q}(\sqrt{6})$ will also work.
\end{example}
