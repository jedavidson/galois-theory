\chapter{Galois correspondence}

In all that follows, let $K/F$ be a finite extension, and write $G = \gal(K/F)$.

\section{Intermediate fields and Galois correspondences}

\begin{definition}
    An \emph{intermediate field} of $K/F$ is a subfield $L$ of $K$ containing $F$, i.e. $F \subseteq L \subseteq K$.
\end{definition}

\begin{example}
    The intermediate fields of $\bb{Q}(\sqrt{2}, \sqrt{3})/\bb{Q}$ are
    \[
        \bb{Q}, \quad \bb{Q}(\sqrt{2}, \sqrt{3}), \quad \bb{Q}(\sqrt{2}), \quad \bb{Q}(\sqrt{3}) \quad \text{and} \quad \bb{Q}(\sqrt{6}).
    \]
\end{example}

\begin{example}
    More generally, if $H$ is a subgroup of $G$, then $F \subseteq K^H$, since by definition every $\sigma \in H$ fixes $F$. So $K^H$ is an intermediate field of $K/F$. In the other direction, if $L$ is an intermediate field of $K/F$, then $H = \gal(K/L)$ is a subgroup of $G$, because $H$ is a subgroup of $\aut{K}$ which fixes $L$ (and so in particular, fixes $F$ as well).
\end{example}

This example describes a kind of correspondence between intermediate fields and subgroups, which we will formalise with the following definition:

\begin{definition}
    A \emph{Galois correspondence} consists of a pair of maps
    \[
        \{\text{intermediate fields } L \text{ of } K/F\}
        \correspondence
        \{\text{subgroups } H \text{ of } \gal(K/F)\},
    \]
    given explicitly by
    \[
        L \stackrel{\phi}{\longmapsto} \gal(K/L) := L \quad \text{and} \quad H \stackrel{\psi}{\longmapsto} K^H := H'.
    \]
\end{definition}

\begin{remark}
    We shall use symbols affixed by $'$ to always indicate images under these two maps. This is not to be confused with the notation for derived groups/commutator subgroups!
\end{remark}

\begin{proposition}
    If $L_1 \subseteq L_2$ are intermediate fields of $K/F$, then $L_1' \supseteq L_2'$. Similarly, if $H_1 \subseteq H_2$ are subgroups of $G$, then $H_1' \supseteq H_2'$. In other words, Galois correspondences are inclusion-reversing.
\end{proposition}

\section{The fundamental theorem of Galois theory}

\begin{theorem}[Fundamental theorem of Galois theory]
\label{thm_fund_thm_of_gal_thy}
    If $K/F$ is a finite Galois extension, then the Galois correspondence maps $\phi, \psi$ are inverse bijections.
   Equivalently, $L'' = L$ and $H'' = H$.
\end{theorem}

\begin{example}
\label{exmp_bad_gal_corresp}
    This is definitely false for non-Galois extensions! Consider, for example, $\bb{Q}(\sqrt[3]{2})/\bb{Q}$. It has two intermediate fields, but the Galois group of this extension was trivial, and hence only has one subgroup.
\end{example}

\begin{corollary}
    Galois extensions have finitely-many intermediate fields.
\end{corollary}

We will prove \cref{thm_fund_thm_of_gal_thy} in stages. The first half of the claim we will prove using the following lemma, which implies that $\phi$ is injective:

\begin{lemma}
\label{lem_int_field_imp_of_fund_gal_thm}
    Let $L$ be an intermediate field of a Galois extension $K/F$, and let $H = L' = \gal(K/L)$. Then $K/L$ is also Galois, and moreover $L = L'' = H'$.
\end{lemma}

\begin{proof}
    Since $K/F$ is Galois, $K$ is the splitting field of some separable polynomial $f$ over $F$, and hence also the splitting field for $f$ over $L$ (or indeed any subfield of $K$). So $K/L$ (and moreover $K/L''$) is Galois. Next, observe that $L \subseteq L''$, since $L$ is fixed by $H$ by definition, so it is sufficient to prove that $[K : L''] \geq [K : L]$. Once again by definition, $L' \subseteq \gal(K/L'')$, so by \cref{cor_card_of_gal_ext_gal_grp}, we have
    \[
        [K : L''] = |\gal(K/L'')| \geq |H| = |\gal(K/L)| = [K : L].
    \]
    Thus $L = L''$.
\end{proof}

\begin{remark}
    This can fail if $K/F$ is not Galois: in \cref{exmp_bad_gal_corresp}, the intermediate field $L = \bb{Q}$ yields $L' = 1$ and $L'' = K \neq L$. (The problem is that there aren't enough subgroups to make $\phi$ injective.)
\end{remark}

\begin{corollary}
    If $K/F$ is Galois with Galois group $G$, then $F = K^G$. That is, the only elements fixed by all automorphisms of $K/F$ are those in the base field $F$.
\end{corollary}

\begin{proof}
    Apply \cref{lem_int_field_imp_of_fund_gal_thm} with $L = F$.
\end{proof}

\begin{example}
    Consider the extension in \cref{exmp_gal_grp_of_cbrt_2}, which is Galois and has Galois group $G \cong S_3$. The nontrivial subgroups of $G$ are cyclic of order 2 or 3 by Lagrange's theorem, giving a Hasse diagram\footnote{Because the Galois correspondence is inclusion-reversing, the lines are directed downwards, so that, for example, $1 \leq \gen{(123}$.} of subgroups
    \begin{figure}[h]
    \captionsetup{justification=centering}
        \[
            \begin{tikzcd}
            	&&& 1 \\
            	\\
            	{\langle (123) \rangle} && {\langle (12) \rangle} && {\langle (13) \rangle} && {\langle (23) \rangle} \\
            	\\
            	&&& G
            	\arrow["3", no head, from=3-1, to=1-4]
            	\arrow["2", no head, from=3-3, to=1-4]
            	\arrow["2", no head, from=3-5, to=1-4]
            	\arrow["2", no head, from=3-7, to=1-4]
            	\arrow["2", no head, from=5-4, to=3-1]
            	\arrow["3", no head, from=5-4, to=3-3]
            	\arrow["3", no head, from=5-4, to=3-5]
            	\arrow["3", no head, from=5-4, to=3-7]
            \end{tikzcd}
        \]
    \end{figure}
    
    \newpage

    By correspondence, the Hasse diagram\footnote{The lines in this diagram are directed upwards as usual.} of intermediate fields is
    \begin{figure}[h]
        \[
            \begin{tikzcd}
            	&&& K \\
            	\\
            	{\bb{Q}(\omega)} && {\bb{Q}(\omega^2 \alpha)} && {\bb{Q}(\omega \alpha)} && {\bb{Q}(\alpha)} \\
            	\\
            	&&& \bb{Q}
            	\arrow["3", no head, from=3-1, to=1-4]
            	\arrow["2", no head, from=3-3, to=1-4]
            	\arrow["2", no head, from=3-5, to=1-4]
            	\arrow["2", no head, from=3-7, to=1-4]
            	\arrow["2", no head, from=5-4, to=3-1]
            	\arrow["3", no head, from=5-4, to=3-3]
            	\arrow["3", no head, from=5-4, to=3-5]
            	\arrow["3", no head, from=5-4, to=3-7]
            \end{tikzcd}
        \]
    \end{figure}
    
    As an example, let us check that $K^{\gen{(23)}} = \bb{Q}(\alpha)$. The subgroup $\gen{(23)}$ certainly fixes $\bb{Q}(\alpha)$, so $\bb{Q}(\alpha) \subseteq K^{\gen{(23)}}$. Now $K/\bb{Q}(\alpha)$ is an extension of prime degree 2, so either $K^{\gen{(23)}} = \bb{Q}(\alpha)$ or $K^{\gen{(23)}} = K$. The latter cannot be true, since $(23)$ does not fix $\omega \alpha$, and so it must be that $K^{\gen{(23)}} = \bb{Q}(\alpha)$. So by \cref{lem_int_field_imp_of_fund_gal_thm}, we have identified all of the nontrivial intermediate fields of $K/\bb{Q}$, which was unclear at first.
\end{example}

\begin{example}
    Consider now the \emph{biquadratic extension} in \cref{exmp_gal_grp_of_biquadratic_ext} with Galois group $G = \gen{\sigma, \tau} \cong \bb{Z}_2 \times \bb{Z}_2$. The Hasse diagram of subgroups, and their corresponding intermediate fields, are thus
    \begin{figure}[h]
        \[
            \begin{tikzcd}
            	& 1 &&& {\mathbb{Q}(\sqrt{2}, \sqrt{3})} \\
            	\\
            	{\langle \sigma \rangle} & {\langle \tau \rangle} & {\langle \sigma\tau \rangle} & {\mathbb{Q}(\sqrt{3})} & {\mathbb{Q}(\sqrt{2})} & {\mathbb{Q}(\sqrt{6})} \\
            	\\
            	& G &&& {\mathbb{Q}}
            	\arrow["2"', no head, from=1-2, to=3-2]
            	\arrow["2"', no head, from=3-2, to=5-2]
            	\arrow["2"', no head, from=1-2, to=3-1]
            	\arrow["2", no head, from=1-2, to=3-3]
            	\arrow["2"', no head, from=3-1, to=5-2]
            	\arrow["2", no head, from=3-3, to=5-2]
            	\arrow["2"', no head, from=1-5, to=3-4]
            	\arrow["2"', no head, from=1-5, to=3-5]
            	\arrow["2", no head, from=1-5, to=3-6]
            	\arrow["2"', no head, from=3-4, to=5-5]
            	\arrow["2"', no head, from=3-5, to=5-5]
            	\arrow["2", no head, from=3-6, to=5-5]
            \end{tikzcd}
        \]
    \end{figure}
\end{example}

Before continuing on to prove the remaining part of \cref{thm_fund_thm_of_gal_thy}, we state and prove a necessary lemma and its corollaries.

\begin{lemma}[Artin]
\label{lem_artin}
    Let $K$ be a field and $H$ a finite group of automorphisms of $K$. Then $[K : K^H] \leq |H|$.
\end{lemma}

\begin{proof}
    Write $H = \{\sigma_1, \sigma_2, \ldots, \sigma_n\}$, where $\sigma_1 = \id_K$. Suppose toward a contradiction that $[K : K^H] > |H|$. This implies that $K$ is a $K^H$-vector space of dimension at least $n + 1$, so we can find $\alpha_1, \alpha_2, \ldots, \alpha_{n + 1} \in K$ which are linearly independent over $K^H$. Consider the system of linear equations
    \[
        A \bm{x} =
        \begin{pmatrix}
            \sigma_1(\alpha_1) & \sigma_1(\alpha_2) & \ldots & \sigma_1(\alpha_{n + 1}) \\
            \sigma_2(\alpha_1) & \sigma_2(\alpha_2) & \ldots & \sigma_2(\alpha_{n + 1}) \\
            \vdots & \vdots & \ddots & \vdots \\
            \sigma_n(\alpha_1) & \sigma_n(\alpha_2) & \ldots & \sigma_n(\alpha_{n + 1}) \\
        \end{pmatrix}
        \begin{pmatrix}
            x_1 \\ x_2 \\ \vdots \\ x_{n + 1}
        \end{pmatrix}
        = \bm{0}
    \]
    over $K$. Since this system has more equations than variables, it must have a nonzero solution $\bm{x} \in K^{n + 1}$. Choose such a solution with the maximal possible number of zero components. We may assume that $x_1 = 1$ by permuting the $\alpha_k$'s and dividing each component of $\bm{x}$ by $x_1$ if necessary. Since $\sigma_1 = \id_K$, the first equation in the system is
    \[
        \alpha_1 + x_2 \alpha_2 + \ldots + x_{n + 1} \alpha_{n + 1} = 0,
    \]
    and by assumption of the linear independence of the $\alpha_k$'s, it cannot be the case that $x_k \in K^H$ for all $k$. Once again, by permuting $\alpha_2, \ldots, \alpha_{n + 1}$ if necessary, we may assume that $x_2 \not\in K^H$, so that there exists some $\sigma \in H$ with $\sigma(x_2) \neq x_2$. Applying this map to the $k$th equation in the system, we get
    \[
        (\sigma \sigma_k)(\alpha_1) \sigma(x_1) + (\sigma \sigma_k)(\alpha_2) \sigma(x_2) + \ldots + (\sigma \sigma_k)(\alpha_{n + 1}) \sigma(x_{n + 1}) = 0.
    \]
    Because $\sigma \sigma_k$ is another element of $H$, applying the map to the entire system simply has the effect of permuting the rows of $A$, so we have $A(\sigma \bm{x}) = \bm{0}$, which implies $\bm{y} = \bm{x} - \sigma(\bm{x})$ is also a solution to this system. In particular, $y_1 = 0$, $y_2 \neq 0$ and $y_k = 0$ if $x_k = 0$, so $\bm{y}$ is a nontrivial solution with more zeroes than $\bm{x}$, a contradiction. So $[K : K^H] \leq |H|$.
\end{proof}

\begin{corollary}
\label{cor_fixed_field_ext_is_gal}
    $K/K^H$ is a finite Galois extension.
\end{corollary}

\begin{proof}
    Finiteness follows by \cref{lem_artin}, so let $\{\alpha_1, \alpha_2, \ldots \alpha_n\}$ be a $K^H$-basis for $K$. Each $\alpha_i$ has minimal polynomial $p_i \in K^H[x]$, and by \cref{thm_aut_min_polys}, we can write
    \[
        p_i(x) = \prod_{\beta} (x - \beta),
    \]
    where $\beta$ ranges over the orbits of $\alpha_i$ under $H$. So $\alpha_i$ is separable (as $p_i$ is irreducible and has distinct roots), and $K$ is the splitting field of the polynomial
    \[
        p(x) = p_1(x) p_2(x) \ldots p_n(x),
    \]
    which is also separable. Thus, $K/K^H$ is Galois.
\end{proof}

\begin{corollary}
\label{cor_subgroup_imp_of_fund_gal_thm}
    We have $\gal(K/K^H) = H$, and $[K : K^H] = |H|$.
\end{corollary}

\begin{proof}
    Clearly $H \subseteq \gal(K/K^H)$, so $|H| \leq |\gal(K/K^H)|$. Since $K/K^H$ is Galois,
    \[
        |\gal(K/K^H)| = [K : K^H] \leq |H|
    \]
    by \cref{lem_artin}, which implies that $H = \gal(K/K^H)$. 
\end{proof}

We can now give the proof of the fundamental theorem.

\begin{proof}[Proof of \cref{thm_fund_thm_of_gal_thy}]
    We already have $L = L''$ by \cref{lem_int_field_imp_of_fund_gal_thm}, and now
    \[
        H'' = \gal(K/H') = \gal(K/K^H) = H
    \]
    by \cref{cor_subgroup_imp_of_fund_gal_thm}.
\end{proof}

\begin{example}
    Let $K = \bb{C}(x)$, and consider the map $\sigma : f(x) \longmapsto f(ix)$. Then $\sigma(x^4) = (ix)^4 = x^4$, so $\bb{C}(x^4) \subseteq K^\sigma$. Also, $\sigma^2(x) = -x$ and $\sigma^4(x) = x$. So $\sigma$ has order 4, which is to say that $G = \gen{\sigma}$ is a cyclic group of automorphisms of $K$ with order 4. So $K/K^G$ is Galois with degree 4, and in fact $K^G = \bb{C}(x^4)$. This gives us the Galois correspondence
    \begin{figure}[h]
        \[
            \begin{tikzcd}
            	1 && {\mathbb{C}(x)} \\
            	\\
            	{\langle \sigma^2 \rangle} && {\mathbb{C}(x^2)} \\
            	\\
            	G && {\mathbb{C}(x^4)}
            	\arrow["2", no head, from=1-1, to=3-1]
            	\arrow["2", no head, from=3-1, to=5-1]
            	\arrow["2", no head, from=1-3, to=3-3]
            	\arrow["2", no head, from=3-3, to=5-3]
            \end{tikzcd}
        \]
    \end{figure}
\end{example}
