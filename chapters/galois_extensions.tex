\chapter{Galois extensions}

Some field extensions are less well-behaved than others. The goal of this chapter is to identify a class of extensions which is ``well-behaved'' enough for Galois theory: so-called \emph{Galois extensions}.

\begin{example}
\label{exmp_non_gal_exts}
    It's instructive to look at some field extensions that, from the point of view of Galois theory, are deficient:
    \begin{enumerate}
        \item We have remarked that as a consequence of $\bb{Q}(\sqrt[3]{2})/\bb{Q}$ not having ``enough'' roots of $x^3 - 2$, its Galois group was trivial. This suggests that field extensions that are also splitting fields of some polynomial may be desirable.
        \item Take $F = \bb{F}_p(t)$, the function field in $t$ over $\bb{F}_p$, and consider adjoining a $p$th root of $t$ to obtain the extension $K = F(t^{1/p})$. The minimal polynomial of $t^{1/p}$ is the (irreducible) polynomial $x^p - t \in F[x]$. But in $K[x]$, 
        \[
            x^p - t = \left(x - t^{1/p}\right)^p,
        \]
        since in characteristic $p$, $(a + b)^p = a^p + b^p$. So while this minimal polynomial is irreducible over $F$, it has a single root with multiplicity in $K$, and once again $\gal(K/F)$ is trivial (because any automorphism of $K/F$ maps roots to roots, so it must map $t^{1/p}$ to itself). This suggests that factoring into distinct linear factors might be desirable too. (Note that this phenomenon is impossible in fields of characteristic zero!)
    \end{enumerate}
\end{example}

\section{Algebraic differentiation}

\begin{definition}
    Let $F$ be a field. Let $f \in F[x]$, say
    \[
        f(x) = \sum_{k = 0}^{n} a_k x^k
    \]
    with $a_k \in F$. We define the \emph{formal derivative of $f$} to be the polynomial
    \[
        f'(x) = \sum_{k = 1}^{n} k a_k x^{k - 1} \in F[x].
    \]
\end{definition}

\begin{remark}
    Alternatively, with $R = F[x, h]/\gen{h^2}$, then we can equivalently define $f'$ to be the unique polynomial such that $f(x + h) - f(x) = f'(x)h$ in $R$.
\end{remark}

\begin{example}
    If $F = \bb{F}_3$ and $f(x) = x^2 + 1 \in F[x]$, then the formal derivative of $f$ is $f'(x) = 2x$, but if instead $f(x) = x^3 + 1$, we get $f'(x) = 3x^2 = 0$!
\end{example}

\begin{proposition}
\label{prop_derivative}
    The following usual properties hold for formal derivatives:
    \begin{enumerate}[label=(\alph*)]
        \item $(f + g)' = f' + g'$, $(\lambda f)' = \lambda f'$ and $(fg)' = f'g + fg'$ for $f, g \in F[x]$ and $\lambda \in F$.
        \item $f \in F[x]$ has a multiple root at $\alpha$ iff $f(\alpha) = f'(\alpha) = 0$.
    \end{enumerate}
\end{proposition}

\begin{proof}
    Exercise.
\end{proof}

\section{Separability}

We make precise the ``no repeated roots'' property mentioned in \cref{exmp_non_gal_exts}.

\begin{proposition}
\label{prop_separability_conditions}
    Let $f \in F[x]$ be irreducible, and $L$ the splitting field for $f$ over $F$. Then the following are equivalent:
    \begin{enumerate}[label=(\alph*)]
        \item $f$ factors into \emph{distinct} linear factors over $L$.
        \item $f'$ is not the zero polynomial in $F[x]$.
        \item If $\alpha$ is a root of $f$ (in some suitable field extension), then
        \[
            \#. \text{ of field homomorphisms } F(\alpha) \to L = [F(\alpha) : F] = \deg{f}.
        \]
    \end{enumerate}
\end{proposition}

\begin{proof}
    Recall that any $\sigma: F(\alpha) \to L$ is determined by $\sigma(\alpha)$, which implies
    \[
        \#. \text{ of field homomorphisms } F(\alpha) \to L
        = \#. \text{ of roots of } f \text{ in } L
        = \deg{f},
    \]
    so (a) and (c) are equivalent.
    
    If (a) holds, then $f$ has no multiple roots. If it was the case that $f'$ was zero, then for any root $\alpha$ of $f$ we would have $f'(\alpha) = 0$, which contradicts \cref{prop_derivative}, so (b) must also hold.
    
    Finally, if (b) holds, then $f'$ is not zero. Now $\gcd(f, f') = 1$, because $f$ is irreducible and $f'$ is of strictly lesser degree than $f$, so $af + bf' = 1$ for some $a, b \in F[x]$. Substituting $x = \alpha$, we get
    \[
        a(\alpha)f(\alpha) + b(\alpha)f'(\alpha) = 1
        \quad \implies \quad
        b(\alpha)f'(\alpha) = 1,
    \]
    and it is clear that we cannot have $f'(\alpha) = 0$, as otherwise $0 = 1$. But this is enough to say that $f$ does not have multiple roots by \cref{prop_derivative}, showing that (a) holds.
\end{proof}

\begin{definition}
\label{def_sep_for_irreducibles}
    If any of the conditions in \cref{prop_separability_conditions} hold, we say that the polynomial $f$ and the element $\alpha$ are \emph{separable} (otherwise, \emph{inseparable}).
\end{definition}

\begin{example}
    We now see $f(x) = x^p - t$ over $F_p(t)$ is inseparable as it has repeated roots, but that $x^2 + 1$ over $\bb{F}_3$ is separable.
\end{example}

\begin{definition}
    A field extension $K/F$ is \emph{separable} if it is generated by separable elements over $F$. A (not necessarily irreducible) polynomial $f \in F[x]$ is \emph{separable} if its irreducible factors are separable in the sense of \cref{def_sep_for_irreducibles}.
\end{definition}

\begin{example}
    Be cautious with this definition: $f(x) = (x - 1)^2 \in \bb{F}_3[x]$ is separable, because the irreducible factors $x - 1$ are separable!
\end{example}

A corollary of the above explains why extensions of $\bb{Q}$, for example, naturally meet the definition of separability:

\begin{corollary}
    If $\fieldchar{F} = 0$, then all extensions of $F$ are separable.
\end{corollary}

\begin{proof}
    Follows from the fact that the only polynomials for which the formal derivative vanishes over such fields are the constant polynomials.
\end{proof}

\section{Normal and Galois extensions}

The property that the first extension lacks in \cref{exmp_non_gal_exts} is \emph{normality}.

\begin{definition}
    A field extension $K/F$ is said to be \emph{normal} if $K$ is the splitting field of some polynomial $f \in F[x]$.
\end{definition}

\begin{remark}
    This definition is technically wrong for infinite extensions, but in this course we will deal primarily in finite extensions (for which it is valid).
\end{remark}

Here is an equivalent definition for normality that appears in some texts. Again, we assume finiteness so that our first definition works.

\begin{proposition}
\label{prop_normal_ext_min_polys_split}
    If $K/F$ is normal, then the minimal polynomial over $F$ of any element in $K$ splits in $K$.
\end{proposition}

\begin{proof}
    Since $K$ is the splitting field of some $f \in F[x]$, write $K = F(\alpha_1, \ldots, \alpha_n)$ where the $\alpha_k$'s are the roots of $f$. Pick any $\alpha \in K$. If $\alpha$ is already in $F$, then its minimal polynomial trivially splits in $K$, so assume that $\alpha \not\in F$. Its minimal polynomial $g \in F[x]$ must therefore be of degree at least 2. Let $\beta$ be some other root of $g$ in a suitable splitting field.
    % Let $L$ be a splitting field for $g$ over $K$, and let $\beta \in L$ be any other root of $g$.
    % Consider $fg$ as a polynomial over $K$, and let $L$ be its splitting field over $K$. Then $g$ splits in $L$ too, so let $\beta \in L$ be any other root of $g$.
    By \cref{prop_field_isom_lifts_to_root_isom}, there is a field isomorphism $\sigma: F(\alpha) \to F(\beta)$ with $\sigma = \id$ on $F$ and $\sigma(\alpha) = \beta$. Note that $f$ splits in $K$ over $F(\alpha)$, and $\sigma f = f$ splits in $K(\beta)$ over $F(\beta)$, so \cref{prop_field_isom_ext_to_spl_field_isom} allows for an extension of $\sigma$ to an isomorphism $\widetilde{\sigma}: K \to K(\beta)$. It follows from the proof of this proposition that $\widetilde{\sigma} = \sigma$ on $F(\alpha)$, and we know that $\sigma = \id$ on $F$, so in particular $\widetilde{\sigma}f = f$. Now for each root $\alpha_k$ of $f$, we have
    \[
        f(\widetilde{\sigma}(\alpha_k)) = (\widetilde{\sigma} f)(\alpha_k) = f(\alpha_k) = 0,
    \]
    so $\widetilde{\sigma}(\alpha_k) \in K$ for each $k$, and hence $\widetilde{\sigma}(\alpha) \in K$. But $\beta = \sigma(\alpha) = \widetilde{\sigma}(\alpha)$, and since $\beta$ was an arbitrarily-chosen root of $g$, this shows that $g$ splits in $K$.
\end{proof}

\begin{exercise}
    Complete the proof of equivalence of these normality definitions by showing that the converse of \cref{prop_normal_ext_min_polys_split} also holds.
\end{exercise}

\begin{definition}
    A normal field extension $K/F$ is \emph{Galois} if, in particular, it is the splitting field of some separable polynomial over $F$.
\end{definition}

\begin{example}
\label{exmp_gal_ext}
    The extension in \cref{exmp_gal_grp_of_cbrt_2} is Galois, because it is the splitting field of $x^3 - 2$, which is separable over $\bb{Q}$.
\end{example}

\begin{theorem}
    Let $\sigma: F \to F'$ be a field homomorphism, and $f \in F[x]$ a separable polynomial. Then if $K, K'$ are splitting fields of $f, \sigma f$ respectively, there are exactly $[K : F]$ field homomorphisms $\sigma_K: K \to K'$ extending $\sigma$.
\end{theorem}

\begin{proof}
    We proceed by induction on $n = [K : F]$. The base case $n = 1$ is trivial, so suppose $n > 1$. Then $f$ has an irreducible factor $g \in F[x]$ of degree $\geq 2$. Fix a root $\alpha \in K$ of $g$, and let $\alpha' \in K'$ be a root of $\sigma g$. Then $K, K'$ are splitting fields for $f$ and $\sigma g$ over $F(\alpha)$ and $F'(\alpha')$ respectively, and by \cref{prop_field_isom_lifts_to_root_isom} there is a unique extension of $\sigma$ to $\widetilde{\sigma}: F(\alpha) \to F'(\alpha')$. Applying induction to the extension $K/F(\alpha)$ of degree $< n$, we get
    \[
        \#. \text{ extensions of } \widetilde{\sigma} \text{ to } \hat{\sigma}: K \to K' = [K : F(\alpha)],
    \]
    and so given that there are $[F(\alpha) : F]$ homomorphisms $\widetilde{\sigma}$ to begin with (each coinciding with a choice of $\alpha'$), we get
    \[
        \#. \text{ field homomorphisms } K \to K' = [K : F(\alpha)] \cdot [F(\alpha) : F] = [K : F],
    \]
    as claimed.
\end{proof}

\begin{corollary}
\label{cor_card_of_gal_ext_gal_grp}
    Applied to $\sigma = \id_{F}$, there are exactly $[K : F]$ automorphisms of $K/F$. In other words, if $K/F$ is Galois, then $|\gal(K/F)| = [K : F]$.
\end{corollary}

\begin{example}
    We gave a non-trivial argument in \cref{exmp_gal_grp_of_cbrt_2} that the Galois group $G$ of $\bb{Q}(\sqrt[3]{2}, e^{2\pi i/3})/\bb{Q}$ is of order 6 (i.e. isomorphic to $S_3$). \cref{cor_card_of_gal_ext_gal_grp} gives us a much simpler alternative argument: this extension is Galois of degree 6, so $G$ is (isomorphic to) a subgroup of $S_3$ with order 6, i.e. $G \cong S_3$.
\end{example}
