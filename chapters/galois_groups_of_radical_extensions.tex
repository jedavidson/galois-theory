\chapter{Galois groups of radical extensions}

Assume for this chapter that $\fieldchar{F} = 0$ for all base fields $F$ we work over. Let $\mu_n$ be the group of $n$th roots of unity in a splitting field $L$ for $x^n - 1$ over $F$. In particular, $\mu_n$ is cyclic of order $n$.

\section{The simple case}

The aim of this section is to generalise the observation that
\[
    \gal\left(\bb{Q}(\sqrt[3]{2}, \zeta_3) / \bb{Q}(\zeta_3)\right) \cong \zn{3}.
\]
In the situations where it is applicable, this will give us a powerful tool for identifying Galois groups as cyclic groups of prime order.

\begin{proposition}
\label{prop_gal_grp_isom_for_root_of_unity_field}
    Let $p$ be prime, and $F$ a field where $x^p - 1$ splits (i.e. $F$ contains all $p$th roots of unity). Let $\alpha \in F$ and $K = F(\sqrt[\uproot{3} p]{\alpha})$ (i.e. adjoin a root of $x^p - \alpha$). Then
    \begin{enumerate}[label=(\alph*)]
        \item $K$ is the splitting field of $x^p - \alpha$.
        \item If $\sqrt[\uproot{3} p]{\alpha} \not\in F$, then we have a well-defined group isomorphism
        \[
            \varphi: G := \gal(K/F) \to \mu_p, \quad \sigma \isomap \frac{\sigma(\sqrt[\uproot{3} p]{\alpha})}{\sqrt[\uproot{3} p]{\alpha}}.
        \]
    \end{enumerate}
\end{proposition}

\begin{proof}~
    \begin{enumerate}[label=(\alph*)]
        \item Clear, as $\mu_p \subseteq F$, and the roots of $x^p - \alpha$ are $\omega \sqrt[\uproot{3} p]\alpha$ for $\omega \in \mu_p$.
        \item We know that $\sigma(\sqrt[\uproot{3} p]{\alpha})$ is a root of $x^p - \alpha$, so $\sigma(\sqrt[\uproot{3} p]{\alpha})/\sqrt[\uproot{3} p]{\alpha}$ is a root of $x^p - 1$, which is to say that it is a $p$th root of unity. Checking that $\varphi$ is a group homomorphism is left an exercise. Because any $\sigma \in G$ is uniquely determined by $\sigma(\sqrt[\uproot{3} p]{\alpha})$, it follows that $\varphi(\sigma) \neq \varphi(\tau)$ whenever $\sigma \neq \tau$, so $\varphi$ is injective. Finally, we have
        \begin{align*}
            \sqrt[\uproot{3} p]{\alpha} \not\in F
            \; \Rightarrow \;
            [K : F] > 1
            \; \Rightarrow \;
            |G| > 1
            \; \Rightarrow \;
            \im{\varphi} > 1,
        \end{align*}
        which forces $\im{\varphi} = \mu_p$, as $\mu_p$ is of prime order. \qedhere
    \end{enumerate}
\end{proof}

\begin{remark}
    The reasoning in part (b) of this proof shows that $[K : F] = p$, and so in particular, $x^p - \alpha$ is irreducible over $F$.
\end{remark}

\section{Solvability}

\begin{definition}
    Let $G$ be a finite group. A \emph{normal chain} of subgroups is a sequence
    \[
        G_0 = 1 \unlhd G_1 \unlhd G_2 \unlhd \cdots \unlhd G_n = G.
    \]
    The quotients $G_{k + 1}/G_k$ are called \emph{factors}. We say that $G$ is \emph{solvable} if there exists such a chain with all factors either trivial or cyclic groups of prime order.
\end{definition}

\begin{remark}
    This definition of solvability only works for finite groups, although the concept may be defined for infinite groups if one only insists that each factor is abelian.
\end{remark}

\begin{example}
\label{exmp_solvable_groups}
    Many groups we have seen before are solvable:
    \begin{enumerate}
        \item 
        There is a nontrivial normal chain $1 \unlhd A_3 \unlhd S_3$, and its factors are
        \[
            S_3/A_3 \cong \zn{2}, \quad A_3/1 \cong A_3 \cong \zn{3},
        \]
        so $S_3$ is solvable.
        
        \item Any cyclic group $\zn{n}$ is solvable for $n \geq 1$. To see this, consider the factorisation $n = p_1 p_2 \cdots p_m$ into (not necessarily distinct) primes $p_k$. Then
        \[
            1 = (p_1 p_2 \cdots p_m) \zn{n} \unlhd (p_1 p_2 \cdots p_{m - 1}) \zn{n} \unlhd \cdots \unlhd \zn{n}
        \]
        is a normal chain, and by the second isomorphism theorem, its factors are
        \[
            \frac{(p_1 p_2 \cdots p_{k - 1}) \zn{n}}{(p_1 p_2 \cdots p_k) \zn{n}} \cong \frac{(p_1 p_2 \cdots p_{k - 1}) \bb{Z}}{(p_1 p_2 \cdots p_k) \bb{Z}} \cong \zn{p_k}. \qedhere
        \]
    \end{enumerate}
\end{example}

\begin{remark}
    The fact that $G_k \unlhd G_{k + 1}$ does not necessarily imply $G_k \unlhd G$. For example, if $G = D_8 = \gen{\tau, \sigma}$, then $\gen{\tau} \unlhd \gen{\tau, \sigma^2}$, but $\gen{\tau} \ntrianglelefteq D_8$.
\end{remark}

\begin{theorem}
    Consider a radical tower
    \[
        F = F_0 \subseteq F_1 \subseteq \cdots \subseteq F_n = K
    \]
    with $F_{k + 1} = F_k(\sqrt[\uproot{3} p_k]{\alpha_k})$ for some $\alpha_k \in F_k$ and $p_k$ prime. Suppose each polynomial $x^{p_k} - 1$ splits in $F[x]$ and that $K/F$ is Galois. Then $G = \gal(K/F)$ is solvable.
\end{theorem}

\begin{proof}
    Using the Galois correspondence, there exists a chain of subgroups
    \[
        G = F_0' \geq F_1' \geq F_2' \geq \cdots \geq F_n' = 1. \tag{$*$}
    \]
    Since $K/F$ is Galois, so is each upper extension $K/F_k$. By \cref{prop_addendum_fund_thm_of_gal_thy}, we have $\gal(K/F_{k + 1}) = F_{k + 1}' \unlhd F_k' = \gal(K/F_k)$, and the factors have the form
    \[
        F_k'/F_{k + 1}' \cong \gal(F_{k + 1}/F_k).
    \]
    By \cref{prop_gal_grp_isom_for_root_of_unity_field}\footnote{with $p = p_k$, $F = F_k$ and $\alpha = \alpha_k$}, each of these factors is either isomorphic to $\zn{p_k}$ or trivial, so the required chain is in fact ($*$).
\end{proof}

\begin{example}
    Consider the biquadratic extension
    \[
        \bb{Q} \subseteq \bb{Q}(\sqrt{2}) \subseteq \bb{Q}(\sqrt{2}, \sqrt{3})
    \]
    with $G = \gal(\bb{Q}(\sqrt{2}, \sqrt{3})/\bb{Q})) \cong K_4$. The normal chain is
    \[
        1 \unlhd \gen{(34)} \unlhd G,
    \]
    with factors $\gen{(34)}/1 \cong G/\gen{(34)} \cong \zn{2}$.
\end{example}
