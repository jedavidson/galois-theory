\chapter{$p$-groups and Sylow theorems}

\section{Revision: group actions}

Let $G$ be a finite group, and $X$ some finite set.

\begin{definition}
    Suppose that we have a homomorphism $\varphi: G \to \perm{X}$. Then we say that $G$ \emph{acts on} $X$ via $g . x = \varphi(g)(x)$. The map $(.) : G \times X \to X$ is called a \emph{group action}.
\end{definition}

\begin{example}
    Any group $G$ acts on itself by conjugation:
    \[
        \operatorname{conj}: G \to \perm{G}, \quad g \longmapsto (h \longmapsto ghg^{-1}).
    \]
\end{example}

\begin{definition}
    For each $x \in X$, the $G$-\emph{orbit} of $x$ is the set
    \[
        G . x := \{g.x : g \in G\}.
    \]
    The \emph{stabiliser} of $x$ in $G$ is the subgroup
    \[
        \stab{G}{x} := \{g \in G: g . x = x\} \leq G.
    \]
\end{definition}

\begin{proposition}~
    \begin{enumerate}[label=(\alph*)]
        \item $X$ is a disjoint union of $G$-orbits.
        \item $|G . x| = [G : \stab{G}{x}]$.
    \end{enumerate}
\end{proposition}

\begin{proof}
    Exercise.
\end{proof}

\begin{definition}
    The \emph{centre} of $G$ is the set
    \[
        Z(G) := \{z \in G: gz = zg \text{ for all } g \in G\}.
    \]
    Equivalently, $z \in Z(G)$ iff $gzg^{-1} = z$ for all $g \in G$ (or $zgz^{-1} = g$).
\end{definition}

\begin{proposition}
    $Z(G) \unlhd G$.
\end{proposition}

\begin{proof}
    Exercise.
\end{proof}

\begin{example}
    The centre can be trivial (e.g. $S_3$) or the full group (e.g. any abelian group). For $D_8 = \gen{\tau, \sigma \mid \tau^2 = 1, \, \sigma^4 = 1, \, \sigma \tau = \tau \sigma^{-1}}$, we have the nontrivial, proper centre $Z(D_8) = \gen{\sigma^2}$.
\end{example}

\begin{definition}
    The set of \emph{fixed points} of $G$ acting on $X$ is
    \[
        X^G := \{x \in X: g . x = x \text{ for all } g \in G\}.
    \]
\end{definition}

\section{$p$-groups and Sylow $p$-subgroups}

Let $p$ be prime.

\begin{definition}
    A group $P$ is a $p$-group if $|P| = p^k$ for some $k \geq 1$.
\end{definition}

\begin{example}
    $D_8$ and $\zn{9}$ are $p$-groups, but $S_3$ is not.
\end{example}

\begin{definition}
    Let $G$ be a finite group, and $P \leq G$ a subgroup. Then $P$ is a \emph{Sylow $p$-subgroup} if $|P| = p^k$ where $p^k$ is the largest power of $p$ dividing $|G|$.
\end{definition}

\begin{definition}
    For $G = S_3$, we have $|G| = 2 \times 3$. The only Sylow 3-subgroup is $A_3$, and the three subgroups generated by a 2-cycle are Sylow 2-subgroups.
\end{definition}

\begin{lemma}
    Let $P$ be a $p$-group acting on a finite set $X$. Then $|X| \equiv |X^p| \pmod{p}$.
\end{lemma}

\begin{proof}
    We see that $X^p$ is the union of single-element $P$-orbits, and all other orbits have more than one element. For a typical orbit $P.x$, $|P.x| = [P : \stab{P}{x}]$, which must be divisible by $p$ since $\stab{P}{x}$ is a proper subgroup and $|P|$ has $p$-power order. So $|X \setminus X^p| \equiv 0 \pmod{p}$.
\end{proof}

\begin{corollary}
    Any $p$-group $P$ has nontrivial centre.
\end{corollary}

\begin{proof}
    Apply the previous lemma to the conjugation action on $P$.
\end{proof}

\begin{corollary}
    $P$ has a normal chain of subgroups with factors $\zn{p}$, so in particular any $p$-group is solvable.
\end{corollary}

\begin{proof}
    Consider $Z(P) \neq 1$. Since $Z(P)$ is abelian, it is solvable, and $Z(P) \unlhd P$. Now $P/Z(P)$ is also a $p$-group, and $|P/Z(P)| < P$, so by induction on the order, we see that $P/Z(P)$ is also solvable. So $P$ itself is solvable by \cref{prop_solvability_of_subgroups_and_quots}, and its factors are abelian $p$-groups. As an exercise, use the structure theorem to reduce these factors to $\zn{p}$.
\end{proof}

\section{Sylow theorems}

Let $p$ be prime.

\begin{theorem}
    Let $G$ be a group of order $p^kq$, where $p \nmid q$. Then
    \begin{enumerate}[label=(\alph*)]
        \item There exists a Sylow $p$-subgroup $P$.
        \item Any $p$-subgroup $H$ of $G$ (i.e. a subgroup of $p$-power order) is contained in a conjugate of $P$. So every Sylow $p$-subgroup is a conjugate of $P$.
        \item The number of Sylow $p$-subgroups is congruent to 1 modulo $p$, and it divides $q$.
    \end{enumerate}
\end{theorem}

\newpage

\begin{proof}~
    \begin{enumerate}[label=(\alph*)]
        \item Let $X = \{S \subseteq G: |S| = p^k\}$. Then $G$ acts on $X$ via $g.S = gS$. We claim that $|X| \equiv q \pmod{p}$, i.e. $|X| \not\equiv 0 \pmod{p}$. In order to identify
        \[
            |X| = \binom{|G|}{p^k} = \binom{p^kq}{p^k},
        \]
        we will consider the polynomial $(x + y)^{p^kq} \in \bb{F}_p[x, y]$. On one hand, $|X|$ is the coefficient of $x^{pkn} y^{p^k(q - 1)}$ by the binomial expansion. On the other hand, $(x + y)^{p^k} = x^{p^k} + y^{p^k}$ over $\bb{F}_p$, so again by the binomial theorem
        \begin{align*}
            (x + y)^{p^k q} = (x^{p^k} + y^{p^k})^q = x^{p^kq} + \ldots + \textcolor{red}{q}x^{p^k} y^{p^k(q - 1)} + y^{p^kq}.
        \end{align*}
        Comparing coefficients, $|X| = q$ in $\bb{F}_p$, i.e. $|X| \equiv q \pmod{p}$. This shows that there must be a $G$-orbit $G.S$ such that $p \nmid |G.S|$. We next claim that if $P := \stab{G}{S}$, then $P$ is a Sylow $p$-subgroup. We know $[G : P] = |G.S|$, so
        \[
            p \nmid \frac{|G|}{|P|} = \frac{p^kq}{|P|},
        \]
        which is to say that $p^k \mid |P|$. Also, $P$ stabilises $S$, so
        \[
            PS = \{gs: g \in P\} \subseteq S,
        \]
        i.e. $|P| = |Ps| \leq |S| = p^k$. This forces $|P| = p^k$, so $P$ is a Sylow $p$-subgroup.
        
        \item Suppose that $H \leq G$ is a $p$-group. Consider the action of $H$ on $G/P$ (which we do not assume is a group) via multiplication on the left. By the previous lemma,
        \[
            q = |G/P| \equiv \left|(G/P)^H\right| \pmod{p}.
        \]
        So $(G/P)^H$ contains a coset $gP$ with $h(gP) = gP$ for all $h \in H$, i.e.
        \begin{align*}
           g^{-1}hgP = P
           &\implies g^{-1}hg \in P \\
           &\implies h \in gPg^{-1} \\
           &\implies H \subseteq gPg^{-1}.
        \end{align*}
    
        \item Omitted. \qedhere
    \end{enumerate}
\end{proof}

\begin{example}
    Let us look at $S_5$. We have $|S_5| = 120 = 2^3 \cdot 3 \cdot 5$. Any subgroup generated by a 5-cycle, e.g. $\gen{(12345)}$, is a Sylow 5-subgroup, which are all conjugate to each other. Similarly, any subgrouo generated by a 3-cycle, e.g. $\gen{(123)}$, is a Sylow 3-subgroup, which are all conjugates of each other. Sylow 2-subgroups have order 8, and an example is any subgroup isomorphic to $D_8$, the symmetry group of a square with vertices labelled 1, 2, 3, 4 (keeping the fifth point labelled 5 fixed); one example is $\gen{(13), (1234)}$. All other isomorphic subgroups may be obtained by relabelling, which is precisely the effect of conjugation in $S_5$. In fact, this gives a descripion of all subgroups of order 8 in $S_5$. Moreover, the second Sylow theorem shows that a subgroup of order 2 or 4 is a subgroup of some $D_8$ isomorph. The subgroups of order 2 are those generated by 2-cycles or a product of 2-cycles, and the subgroups of order 4 are those generated by 4-cycles, two 2-cycles or two products of 2-cycles.
\end{example}

\begin{exercise}
    Count how many Sylow 2-subgroups there are to verify the third Sylow theorem.
\end{exercise}

\section{An application: proving the fundamental theorem of algebra}

\begin{block}
    This may be a bit unpolished.
\end{block}

Many proofs exist already, relying on either complex analysis or algebraic topology. We aim to give a proof largely (but not completely) using just Galois theory.

\begin{theorem}
    Any nonconstant $f \in \bb{C}[x]$ has a root in $\bb{C}$.
\end{theorem}

\begin{proof}
    We assume the following analytic facts, courtesy of the intermediate value theorem:
    \begin{enumerate}[label=(\alph*)]
        \item Any odd polynomial in $\bb{R}[x]$ has a real root.
        \item Any positive real number has a square root.
    \end{enumerate}
    In particular, every complex number has a complex square root, as
    \[
        \sqrt{a + ib} = x + iy \implies (x + iy)^2 = a + ib,
    \]
    and we can now compare coefficients to find that
    \[
        \begin{cases}
            x^2 - y^2 = a \\
            2xy = b
        \end{cases}
        \implies x^4 - ax^2 - \frac{b^2}{4} = 0.
    \]
    This is a quadratic in $x^2$, with solution
    \[
        x^2 = \frac{a \pm \sqrt{a^2 + b^2}}{2}.
    \]
    By (b), the right-hand side has a positive square root (take $+$), so one can solve for $x$ and then $y$. Now, suppose by contradiction that $f \in \bb{C}[x]$ is irreducible with $\deg{f} \geq 2$. The splitting field $K$ of $f$ over $\bb{C}$ admits a finite extension $K/\bb{C}$ of degree at least 2. We will work over $\bb{R}$ instead (i.e. $[K : \bb{R}] \geq 4$). Replacing $K$ with its Galois closure if necessary, we may assume $K/\bb{R}$ is Galois. Let $G = \gal(K/\bb{R})$, and let $P$ be a Sylow 2-subgroup of $G$. Now $K/K^P$ is Galois with Galois group $P$, and since $\fieldchar{R} = 0$, the primitive element theorem implies $K^P = \bb{R}(\alpha)$ for some $\alpha \in K$. Note that
    \[
        [K^P : \bb{R}] = \frac{|K : \bb{R}|}{|K : K^P|} = \frac{|G|}{|P|},
    \]
    which must be odd. This implies that the minimal polynomial $g \in \bb{R}[x]$ of $\alpha$ has odd degree. By the first assumption, $g$ has a real root, which is to say that $g$ is an irreducible polynomial with a linear factor. But this means that $g$ is itself a linear polynomial, so $K^P = \bb{R}$, and hence $\gal(K/\bb{R}) = P$. Next, we know that all $p$-groups are solvable, so there is a normal chain of subgroups
    \[
        P = P_0 \unrhd P_1 \unrhd P_2 \unrhd \cdots \unrhd P_n = 1,
    \]
    with abelian factors $P_i/P_{i + 1} \cong \zn{2}$. The corresponding fields give a tower of quadratic extensions
    \[
        \bb{R} \subseteq K^{P_1} \subseteq K^{P_2} \subseteq \cdots \subseteq K.
    \]
    Applying the quadratic formula with the second assumption, we have
    \[
        K^{P_1} \cong \bb{R}(\sqrt{< 0}) \cong \bb{C}.
    \]
    But $K^{P_2}$ is now a quadratic extension of $\bb{C}$, which is impossible (by the second assumption).
\end{proof}

\begin{remark}
    The ``fact" that there are at least 2 quadratic extensions of $\bb{R}$ at the end follows from the ``fact" that $[K : \bb{R}] \geq 4$.
\end{remark}
