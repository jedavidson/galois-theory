\chapter{Revision: field extensions, algebraics and transcendentals}

We first recap some key concepts of fields from MATH3711.

\begin{definition}
    Let $F$ be a field. A field $K$ such that $F \subseteq K$ is called a \emph{field extension} of $F$, and we write $K/F$.
\end{definition}

\begin{remark}
    This is not to be viewed as a quotient! Also, some authors (e.g. Stewart) define a field extension to be a monomorphism of fields $\varphi: F \to K$. This is slightly more general than our definition and can be useful at times.
\end{remark}

\begin{example}
    Some field extensions are $\bb{Q}(\sqrt{2})/\bb{Q}$, $\bb{C}/\bb{R}$ and $F/F$.
\end{example}

\begin{definition}
    Let $K/F$ be a field extension and $\alpha \in K$. Then $\alpha$ is \emph{algebraic} over $F$ if $f(\alpha) = 0$ for some (nonzero!) $f \in F[x]$.
\end{definition}

\begin{example}
    $\sqrt{2}$ is algebraic over $\bb{Q}$, since $x^2 - 2 = 0$.
\end{example}

\begin{definition}
    Let $\alpha \in F$. The \emph{minimal polynomial} of $\alpha$ is the polynomial $p \in F[x]$ of minimal degree such that $p(\alpha) = 0$.
\end{definition}

\begin{remark}
    Some authors require $p$ to be monic, because the minimal polynomial is only unique up to a constant factor. It may also be referred to as the irreducible polynomial of $\alpha$.
\end{remark}

\begin{example}
    \begin{enumerate}
        \item The minimal polynomial of $\sqrt{2}$ in $\bb{Q}$ is $x^2 - 2$.
        \item The minimal polynomial of $i$ in $\bb{R}$ is $x^2 + 1$.
        \item The minimal polynomial of $\pi$ in $\bb{R}$ is $x - \pi$.
    \end{enumerate}
\end{example}

\begin{definition}
    Let $K/F$ be a field extension. The \emph{simple extension} of $F$ generated by $\alpha$ is $F(\alpha)$, the smallest subfield of $K$ containing both $F$ and $\alpha$.
\end{definition}

\begin{theorem}[Structure theorem for simple extensions, algebraic]
\label{struct_thm_simple_exts_alg}
    If $\alpha \in K$ is algebraic over $F$ with minimal polynomial $p \in F[x]$, then there is an isomorphism of fields $F(\alpha) \cong F[x] / \gen{p}$ given explicitly by
    \[
        g(x) + \gen{p} \isomap g(\alpha).
    \]
\end{theorem}

\begin{example}
\label{struct_thm_simple_exts_alg_example}
    Let $\alpha = \sqrt[3]{2}$. Its minimal polynomial over $\bb{Q}$ is $x^3 - 2$, so by \cref{struct_thm_simple_exts_alg}, we have $\bb{Q}(\alpha) \cong \bb{Q}[x]/\gen{x^3 - 2}$. So every element of $\bb{Q}(\alpha)$ can be written \emph{uniquely} in the form
    \[
        u_0 + u_1 \alpha + u_2 \alpha^2 = u_0 + u_1 \sqrt[3]{2} + u_2 \sqrt[3]{4}
    \]
    for some $u_k \in \bb{Q}$, via the isomorphism $x \isomap \alpha$.
\end{example}

\begin{exercise}
    Repeat \cref{struct_thm_simple_exts_alg_example} for $\bb{Q}(i)$.
\end{exercise}

\begin{corollary}
    The degree of the extension $F(\alpha)/F$ is $[F(\alpha) : F] = \deg{p}$.
\end{corollary}

\begin{definition}
    An element $\alpha \in K$ which is not algebraic over $F$ (that is, the root of no polynomial in $F[x]$) is called \emph{transcendental} over $F$.
\end{definition}

\begin{theorem}[Structure theorem for simple extensions, transcendental]
    If $\alpha \in K$ is transcendental, then there is an isomorphism of fields $F(\alpha) \cong F(x)$, the field of rational functions over $F$, via the isomorphism $x \isomap \alpha$.
\end{theorem}
