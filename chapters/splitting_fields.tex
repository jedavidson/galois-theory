\chapter{Splitting fields}

\section{Definitions}

\begin{definition}
    Suppose $f \in F[x]$. An extension $K$ of $F$ is a \emph{splitting field for} $f$ \emph{over} $F$ if
    \begin{enumerate}[label=(\alph*)]
        \item $f(x)$ splits (i.e. factors) into linear factors in $K[x]$ (i.e. $K$ includes all of the roots of $f$), and
        \item $K = F(\alpha_1, \ldots, \alpha_n)$, where the $\alpha_k$ are the roots of $f$ in $K$.
    \end{enumerate}
\end{definition}

\begin{remark}
    Condition (b) does not subsume (a) because of the restriction that $\alpha_k \in K$. Consider $\bb{Q}(\sqrt[3]{2})$ and $f(x) = x^3 - 2$, whose only root in $\bb{Q}(\sqrt[3]{2})$ is $\sqrt[3]{2}$. Then $f$ does not split into linear factors over $\bb{Q(\sqrt[3]{2})}[x]$.
\end{remark}

\begin{example}
    $\bb{C}$ is a splitting field for $x^2 + 1$ over $\bb{R}$, but not over $\bb{Q}$ because $\bb{C} \neq \bb{Q}(i, -i)$. This tells us that, in some sense, splitting fields are the smallest fields that contain the roots of the specified polynomial.
\end{example}

\begin{example}
\label{exmp_spl_field_of_cbrt_2}
    Let $f(x) = x^3 - 2$ over $\bb{Q}$. Let $\alpha = \sqrt[3]{2}$, $\omega = e^{2\pi i/3}$. The roots of $f$ are $\alpha, \omega \alpha$ and $\omega^2 \alpha$, so $K = \bb{Q}(\alpha, \omega \alpha, \omega^2 \alpha)$ is a splitting field for $f$ over $\bb{Q}$. In fact, $K = \bb{Q}(\alpha, \omega)$, because
    \begin{itemize}
        \item[($\subseteq$)] $\alpha, \omega \alpha, \omega^2 \alpha \in \bb{Q}(\alpha, \omega)$ (since fields are closed under multiplication), and
        \item[($\supseteq$)] $\alpha, \omega \in K$ (since $\omega = \omega \alpha / \alpha$, and fields are closed under non-zero division).
    \end{itemize}
\end{example}

\begin{corollary}
    If $\alpha_1, \ldots, \alpha_n$ are roots of $f \in F[x]$ in a splitting field $K$, then any $\sigma \in \gal(K/F)$ permutes these roots. This implies that we have a monomorphism $\gal(K/F) \inclusion \perm \{\alpha_1, \ldots, \alpha_n\} \cong S_n$. That is,
    \begin{itemize}
        \item any $\sigma \in \gal(K/F)$ can be identified with a permutation, and
        \item the map $\gal(K/F) \to \perm \{\alpha_1, \ldots, \alpha_n\}$ is a group homomorphism, and
        \item this map is injective (because $\sigma$ is determined by its action on the roots, and $K$ is generated by those roots over $F$).
    \end{itemize}
\end{corollary}

This corollary allows us to identify (in terms of isomorphism) $\gal(K/F)$ as a subgroup of $S_n$, which is perhaps the way Galois himself thought of Galois groups without the machinery of field extensions.

\begin{example}
    Take $\bb{C}$ as the splitting field of $x^2 + 1$ over $\bb{R}$. Then we have a monomorphism
    \[
        \gal(\bb{C}/\bb{R}) \inclusion \perm\{i, -i\} \cong S_2.
    \]
    This tells us, more formally, that there are only 2 $\bb{R}$-automorphisms of $\bb{C}$: the identity map (do nothing) and the conjugation map (interchange $\pm i$).
\end{example}

\begin{example}
\label{exmp_gal_grp_of_biquadratic_ext}
    Let $K = \bb{Q}(\sqrt{2}, \sqrt{3})$, which is a splitting field for $(x^2 - 2)(x^2 - 3)$ over $\bb{Q}$ (though this is not the only choice of polynomial here by any means). Then any $\sigma \in \gal(K/\bb{Q})$ permutes its roots
    \[
        \alpha_1 = \sqrt{2},
        \quad
        \alpha_2 = -\sqrt{2},
        \quad
        \alpha_3 = \sqrt{3},
        \quad \text{and} \quad
        \alpha_4 = -\sqrt{3}.
    \]
    There are 4 legal permutations of these roots: $1$, $\sigma = (12)$, $\tau = (34)$ and $\tau \sigma = (12)(34)$. Permutations such as $(13)$ are illegal, because by applying \cref{prop_f_homs_map_roots_to_roots} to $x^2 - 2$, we see that any automorphism in $\gal(K/F)$ must map $\sqrt{2}$ to one of $\pm \sqrt{2}$ (and similarly for $\sqrt{3}$). All told, then, we have
    \[
        \gal(K/\bb{Q}) = \{1, \sigma, \tau, \tau \sigma\} = \gen{\sigma, \tau} \cong \zn{2} \times \zn{2}.
    \]
    (This is also isomorphic to the \emph{Klein four-group}, the smallest acyclic group.)
\end{example}

\section{Existence and uniqueness of splitting fields}

A yet unaddressed question is how one knows that a splitting field exists for some $f \in F[x]$ in the first place. This is certainly fine if $F \subseteq \bb{C}$, because $\bb{C}$ is algebraically-closed (c.f. the fundamental theorem of algebra). As it turns out though, this is always true: there is always such a splitting field when $F$ is any field, and more than that it is unique up to isomorphism. The aim of this section is to prove this.

A natural first question to ask is how we can find a field extension that includes at least one root of some given polynomial. This is the content of the following proposition:

\begin{proposition}
\label{prop_exists_field_ext_containing_root}
    Let $F$ be a field, and $p \in F[x]$ irreducible. Then
    \begin{enumerate}[label=(\alph*)]
        \item $K = F[x]/\gen{p}$ is a field extension of $F$, and
        \item $K = F(\alpha)$, where $\alpha = x + \gen{p}$ is a root of $p$.
    \end{enumerate}
\end{proposition}

We defer the proof for a moment to look at an example.

\begin{example}
\label{exmp_quot_const_of_c}
    Let $F = \bb{R}$, $p(x) = x^2 + 1$. Then $p$ is irreducible in $F[x]$. Then
    $K = F[x]/\gen{x^2 + 1}$ is a field (as $\gen{x^2 + 1}$ is a maximal ideal), and $\bb{R}$ embeds into $K$ as the constant maps (i.e. $\alpha \in R$ corresponds to the constant map $x \longmapsto \alpha$). So $K$ is an extension of $\bb{R}$ containing $\alpha = x + \gen{x^2 + 1}$, and
    \begin{align*}
        p(\alpha)
        &= (x + \gen{x^2 + 1})^2 + (1 + \gen{x^2 + 1}) \\
        &= (x^2 + \gen{x^2 + 1}) + (1 + \gen{x^2 + 1}) \\
        &= x^2 + 1 + \gen{x^2 + 1} \\
        &= \gen{x^2 + 1},
    \end{align*}
    which is zero in $K$. With a priori knowledge of the existence of $\bb{C}$, we can see that really $K \cong \bb{C}$, and though it may not seem like the same thing at all, this is morally what is really going on in the ``invention'' of $i$ as the solution to $x^2 + 1$ that one may have seen when introduced to $\bb{C}$.
\end{example}

\begin{proof}[Proof of \cref{prop_exists_field_ext_containing_root}]
    Since $p$ is irreducible, $K$ is a field (as $\gen{p}$ is a maximal ideal). By identifying elements of $F$ with the corresponding constant map in $F[x]$, we thus have a composite homomorphism of fields
    \[
        F \inclusion F[x] \stackrel{\pi}{\longrightarrow} F[x]/\gen{p} = K.
    \]
    Each component above is injective: the aforementioned embedding of $F$ into $F[x]$ is clearly injective, as is the quotient morphism $\pi: F[x] \to F[x]/\gen{p}$. So $K$ is a field extension of $F$. Now, if $p(x) = \sum_{i = 1}^{n} a_k x^k$, then as in \cref{exmp_quot_const_of_c} we see that
    \[
        p(\alpha) = \ldots = p(x) + \gen{p} = \gen{p},
    \]
    which is zero in $K$. So $\alpha$ is indeed a root of $p$.
\end{proof}

\begin{remark}
    Here we are using the alternative interpretation of what a field extension is, namely that it is a monomorphism of fields. A more slick argument that this map $F \to K$ is injective is that in general, any homomorphism of fields is automatically injective. The proof of this fact amounts to an application of basic ring properties and arithmetic and is left as an exercise.
\end{remark}

The next result details how an isomorphism of fields can be extended to an isomorphism of extension fields containing roots of a polynomial, all while respecting the original isomorphism between the base fields.

\begin{proposition}
\label{prop_field_isom_lifts_to_root_isom}
    Suppose $\sigma: F \to F'$ is a field isomorphism. Let $p \in F[x]$ be irreducible, and $\alpha, \alpha'$ roots of $p$ and $\sigma p$ respectively (in suitable extension fields of $F$ and $F'$). Then $\sigma$ lifts into a unique field isomorphism $\widetilde{\sigma}: F(\alpha) \to F'(\alpha')$ such that $\sigma = \widetilde{\sigma}$ on $F$, and $\widetilde{\sigma}(\alpha) = \alpha'$.
\end{proposition}

\begin{proof}
    With $d = \deg{p}$, this isomorphism is
    \[
        \sum_{k = 1}^{d - 1} a_k \alpha^k \; \stackrel{\widetilde{\sigma}}{\longmapsto} \; \sum_{k = 1}^{d - 1} \sigma(a_k) (\alpha')^k.
    \]
    Proving that this is sufficient is an exercise. Alternatively, one can construct the isomorphism
    \[
        F(\alpha) \cong F[x]/\gen{p} \isomap F'[x]/\gen{\sigma p} \cong F'(\alpha')
    \]
    via the lifted map $\sigma: F[x] \to F'[x]$ obtained by \cref{prop_ring_hom_to_poly_ring_hom}.
\end{proof}

\begin{example}
\label{exmp_gal_grp_of_cbrt_2}
    Let $\beta = \sqrt[3]{2}$, $\omega = e^{2\pi i/3}$. We saw that the splitting field of $x^3 - 2$ is $K = \bb{Q}(\beta, \beta \omega, \beta \omega^2) = \bb{Q}(\beta, \omega)$, and that we have the monomorphism
    \[
        G = \gal(K/\bb{Q}) \inclusion S_3
    \]
    as permutation of the roots of $p$. It turns out that this is actually an isomorphism in this case. To see this, write the roots as
    \[
        \alpha_1 = \beta,
        \quad
        \alpha_2 = \beta \omega
        \quad \text{and} \quad
        \alpha_3 = \beta \omega^2,
    \]
    so that the complex conjugation map $\tau$ corresponds to the transposition $(23)$. Applying \cref{prop_field_isom_lifts_to_root_isom} with $\sigma = \id_\bb{Q}$, $F = F' = \bb{Q}$, and $\alpha = \beta$, $\alpha' = \beta \omega$, we get an isomorphism $\widetilde{\sigma}: \bb{Q}(\beta) \to \bb{Q}(\beta \omega)$ with $\widetilde{\sigma}(\beta) = \beta \omega$. Applying \cref{prop_field_isom_lifts_to_root_isom} again, this time with $F = \bb{Q}(\beta)$, $F' = \bb{Q}(\beta \omega)$, $\widetilde{\sigma}$ playing the role of $\sigma$, and $\alpha = \alpha' = \omega$ as roots of $x^2 + x + 1$, we get a map
    \[
        \hat{\sigma} : \bb{Q}(\beta)(\omega) \to \bb{Q}(\beta \omega)(\omega)
    \]
    which sends $\alpha_1 \mapsto \alpha_2$, $\alpha_2 \mapsto \alpha_3$ and $\alpha_3 \mapsto \alpha_1$, which is the 3-cycle $(123)$. So $G$ is a subgroup containing $(12)$ and $(123)$, which implies $G \cong S_3$ exactly.
\end{example}

The last useful result we will need is that field isomorphisms extend to isomorphisms of splitting fields.

\begin{proposition}
\label{prop_field_isom_ext_to_spl_field_isom}
    Let $f \in F[x]$, $\sigma: F \to F'$ be a field isomorphism and $K, K'$ splitting fields of $f$ and $\sigma f$ respectively. Then $\sigma$ extends to an splitting field isomorphism $\widetilde{\sigma}: K \to K'$.
\end{proposition}

\begin{proof}
    The claim is trivial if $\deg{f} = 1$, so assume $\deg{f} \geq 2$. Then $f$ has an irreducible factor $g \in F[x]$ with $\deg{g} \geq 2$. Let $\alpha_1, \alpha_1'$ be roots of $g, \sigma g$ in $K, K'$ respectively. By \cref{prop_field_isom_lifts_to_root_isom}, there is an isomorphism $\widetilde{\sigma}_1: F(\alpha_1) \to F'(\alpha_1')$ with $\sigma = \widetilde{\sigma}_1$ on $F$. Repeating this process until $f$ no longer has irreducible factors of degree $\geq 2$ via induction, we obtain a field isomorphism
    \[
        \widetilde{\sigma} = \widetilde{\sigma}_k: F(\alpha = \alpha_1, \alpha_2, \ldots, \alpha_k) \to F'(\alpha' = \alpha_1', \alpha_2', \ldots, \alpha_k')
    \]
    with $\sigma = \widetilde{\sigma}$ on $F$. Now $f$ splits in $F(\alpha_1, \ldots, \alpha_k)$, and $\sigma f$ splits in $F'(\alpha_1', \ldots, \alpha_k')$, so $K = F(\alpha_1, \ldots, \alpha_k)$ and $K' = F'(\alpha_1', \ldots, \alpha_k')$, so $\widetilde{\sigma}$ is the required isomorphism of splitting fields.
    % \begin{figure}[h]
    %     \[
    %         \begin{tikzcd}
    %         	F && {F'} \\
    %         	\\
    %         	{F(\alpha)} && {F'(\alpha')} \\
    %         	\\
    %         	K && {K'}
    %         	\arrow["{\sigma}", from=1-1, to=1-3]
    %         	\arrow[hook, from=1-1, to=3-1]
    %         	\arrow[hook', from=1-3, to=3-3]
    %         	\arrow["{\sim \text{ via prop.}}", from=3-1, to=3-3]
    %         	\arrow[hook, from=3-1, to=5-1]
    %         	\arrow[hook', from=3-3, to=5-3]
    %         	\arrow["{\widetilde{\sigma} \text{ via ind.}}", from=5-1, to=5-3]
    %         \end{tikzcd}
    %     \]
    % \caption{Diagram of required construction for \cref{prop_field_isom_ext_to_spl_field_isom}}
    % \end{figure}
\end{proof}

We can now prove the existence and uniqueness theorem for splitting fields.

\begin{theorem}
\label{thm_exist_uniq_of_spl_field}
    Let $F$ be a field and $f \in F[x]$ be non-constant. Then
    \begin{enumerate}[label=(\alph*)]
        \item there exists a splitting field for $f$ over $F$, and
        \item any two splitting fields for $f$ over $F$ are isomorphic.
    \end{enumerate}
\end{theorem}

\begin{proof}~
    \begin{enumerate}[label=(\alph*)]
        \item We proceed by induction on $n = \deg{f}$. The claim is obvious when $n = 1$, because $F$ is a splitting field in this case. Now suppose $n > 1$, and that $g \in F[x]$ is an irreducible factor of $f$ (possibly $f$ itself). Assume that there is a splitting field for any polynomial of degree $n - 1$ over $F$. By \cref{prop_exists_field_ext_containing_root}, there is a simple field extension $F(\alpha)/F$ containing a root $\alpha$ of $g$. By the factor theorem, we can thus write $f(x) = (x - \alpha) h(x)$, where $h \in F(\alpha)[x]$ is of degree $n - 1$. By inductive hypothesis, there is a splitting field $K$ for $h$ over $F(\alpha)$, and moreover this field is also a splitting field for $f$ over $F$, so we are done.
        \item Follows from \cref{prop_field_isom_ext_to_spl_field_isom} by letting $F = F'$ and $\sigma = \id_F$. \qedhere
    \end{enumerate}
\end{proof}

\begin{remark}
\label{rem_spl_field_isom_fixes_base}
    The \cref{prop_field_isom_ext_to_spl_field_isom} used in (b) also implies that any isomorphism of splitting fields for $f$ over $F$ must fix $F$.
\end{remark}

We end off by discussing a purely algebraic construction of the splitting field in \cref{exmp_spl_field_of_cbrt_2}, i.e. one which doesn't assume the existence of $\bb{R}$ and $\bb{C}$.

\begin{example}
    The polynomial $x^3 - 2$ is irreducible over $\bb{Q}$ (as can be seen by Eisenstein's criterion et al.), so using \cref{prop_exists_field_ext_containing_root} we can adjoin a root $\beta$ by considering the extension
    \[
        \bb{Q}[x]/\gen{x^3 - 2} \cong \bb{Q}(\beta),
    \]
    where $\beta = x + \gen{x^3 - 2}$. Over this extension, we have the factorisation
    \[
        x^3 - 2 = (x - \beta)(x^2 + \beta x + \beta^2),
    \]
    and the polynomial $x^2 + \beta x + \beta^2$ is in fact irreducible over $\bb{Q}(\beta)$. To see this, suppose not, i.e. suppose there is a root $\gamma \in \bb{Q}(\beta)$. Then
    \[
        \gamma^2 + \beta \gamma + \beta^2 = 0,
    \]
    and because $\beta \neq 0$, we can write $\omega = \gamma / \beta$ and obtain
    \[
        \omega^2 + \omega + 1 = 0.
    \]
    So $\omega$ is a root of the irreducible polynomial $x^2 + x + 1$ (again, by an indirect application of Eisenstein's criterion). Moreover, we have the tower of extensions $\bb{Q} \subseteq \bb{Q}(\omega) \subseteq \bb{Q}(\beta)$, since $\omega \in \bb{Q}(\beta)$. This is a contradiction though, because $[\bb{Q}(\omega) : \bb{Q}] = 2$ does not divide $[\bb{Q}(\beta) : \bb{Q}] = 3$. We now adjoin a root $\gamma$ of $x^2 + \beta x + \beta^2$ (and hence $x^3 - 2$) to $\bb{Q}(\beta)$ by \cref{prop_exists_field_ext_containing_root} to obtain
    \[
        L := \bb{Q}(\beta)[x]/\gen{x^2 + \beta x + \beta^2} \cong \bb{Q}(\beta)(\omega) \cong \bb{Q}(\beta, \gamma),
    \]
    where $\gamma = x + \gen{x^2 + \beta x + \beta^2}$. Over $L$, we have the factorisation
    \[
        x^3 - 2 = (x - \beta)(x - \gamma)(x - (-(\beta + \gamma))).
    \]
    To give this factorisation in a more familiar form, we noted previously that $\omega = \gamma/\beta$ satisfies $\omega^2 + \omega + 1 = 0$ and $\gamma = \beta \omega$, so
    \[
        - (\beta + \gamma) = - (\beta + \beta \omega) = - \beta (1 + \omega) = \beta\omega^2.
    \]
    The factorisation above may therefore be written
    \[
        x^3 - 2 = (x - \beta)(x - \beta \omega)(x - \beta \omega^2),
    \]
    and its splitting field is $\bb{Q}(\beta, \omega)$.
\end{example}

\begin{remark}
    This example does not lend itself to an algorithm for algebraically computing splitting fields though: we were quite lucky in that the non-linear factor $x^2 + \beta x + \beta^2$ over the partial splitting field $\bb{Q}(\beta)$ could be proven irreducible by our ad-hoc (and non-constructive!) reasoning, but it is quite hard to know or tell whether it will be irreducible in general.
\end{remark}
